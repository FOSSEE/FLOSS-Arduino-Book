\chapter {Interfacing a Light Dependent Resistor}
\thispagestyle{empty}
\label{ldr}

\newcommand{\LocLDRfig}{\Origin/user-code/ldr/figures}
\newcommand{\LocLDRscicode}{\Origin/user-code/ldr/scilab}
\newcommand{\LocLDRscibrief}[1]{{\tt
    \seqsplit{Origin/user-code/ldr/scilab/#1}},
see \fnrefp{fn:file-loc}}
\newcommand{\LocLDRardcode}{\Origin/user-code/ldr/arduino}
\newcommand{\LocLDRardbrief}[1]{{\tt
    \seqsplit{Origin/user-code/ldr/arduino/#1}},
see \fnrefp{fn:file-loc}}

%%%%%%%%%python
\newcommand{\LocLDRpycode}{\Origin/user-code/ldr/python}
\newcommand{\LocLDRpybrief}[1]{{\tt \seqsplit{%
    Origin/user-code/ldr/python/#1}}, see \fnrefp{fn:file-loc}}
%%%%%%python

%%%%%%%%%julia starts
\newcommand{\LocLDRjuliacode}{\Origin/user-code/ldr/julia}
\newcommand{\LocLDRjuliabrief}[1]{{\tt \seqsplit{%
    Origin/user-code/ldr/julia/#1}}, see \fnrefp{fn:file-loc}}
%%%%%%julia ends

%%%%%%OpenModelica Starts
\newcommand{\LocLDROpenModelicacode}{\Origin/user-code/ldr/OpenModelica}  %added for OpenModelica
\newcommand{\LocLDROpenModelicabrief}[1]{{\tt \seqsplit{%
    Origin/user-code/led/OpenModelica/#1}}, see \fnrefp{fn:file-loc}} % added for OpenModelica

%%%%%OpenModelcia Ends


A Light Dependent Resistor (LDR) or Photoresistor is a light sensitive
semiconductor device whose resistance varies with the variation in the
intensity of light falling on it. As the intensity of the incident
light increases, resistance offered by the LDR decreases. Typically,
in dark, the resistance offered by an LDR is in the range of a few
mega ohms. With the increase in light intensity, the resistance
reduces to as low as a few ohms.   

An LDR is widely used in camera shutter control, light intensity
meters, burglar alarms, street lighting control, automatic emergency
lights, etc. In this chapter we shall interface an LDR with the
\arduino\ board.  

\section{Preliminaries}
A typical LDR and its symbolic representation are shown in
\figref{fig:ldr} and \figref{fig:ldrsym} respectively. The shield
provided with the kit has an LDR mounted on it.  The LDR mounted on
the shield looks exactly like the picture in \figref{fig:ldr},
although, the picture looks a lot larger.  This LDR is connected
to the analog pin 5 of the \arduino\ board. The connections for this
experiment are shown in \figref{fig:ldrconn}. However, the user
doesn't need to connect any wire or component explicitly.

\begin{figure}
\centering
\subfloat[Pictorial representation of an LDR]{
\includegraphics[width=\smfig]{\LocLDRfig/ldr.jpg}
\label{fig:ldr}} \hfill
\subfloat[Symbolic representation of an LDR]{
\includegraphics[width=\smfig]{\LocLDRfig/ldr_sym.png}
\label{fig:ldrsym}}
\caption{Light Dependent Resistor}
\end{figure}

\begin{figure}
\centering
\includegraphics[width=\smfig]{\LocLDRfig/ldr-conn.png}
\caption{Internal connection diagram for the LDR on the shield}
\label{fig:ldrconn}
\end{figure}

The LDR mounted on the shield is an analog sensor. Hence, the analog voltage, corresponding to the changing resistance, across its terminals needs to be digitized before being sent to the computer. This is taken care of by an onboard Analog to Digital Converter (ADC) of ATmega328 microcontroller on the \arduino\
board. ATmega328 has a 6-channel, 0 through 5, 10 bit ADC. Analog pin
5 of the \arduino\ board, to which the LDR is connected, corresponds
to channel 5 of the ADC.  As there are 10 bits, 0-5V readings from LDR
are mapped to the ADC values from 0 to 1023. 

LDR is a commonly available sensor in the market. It costs about
Rs. 100. There are multiple manufacturers which provide commercial
LDRs.  Some examples are VT90N1 and VT935G from EXCELITAS TECH, and
N5AC501A085 and NSL19M51 from ADVANCED PHOTONIX. 

\section{Connecting an LDR with \arduino\ using a breadboard}
This section is useful for those who either don't have a shield or don't want to use the shield
for performing the experiments given in this chapter. 

A breadboard is a device for holding the components of a circuit and connecting 
them together. We can build an electronic circuit on a breadboard without doing any 
soldering. To know more about the breadboard and other electronic components, 
one should watch the Spoken Tutorials on Arduino as published on
{\tt https://spoken-tutorial.org/}. Ideally, one should go through all the
tutorials labeled as Basic. However, we strongly recommend the readers should
watch the fifth and sixth tutorials, i.e., {\tt First Arduino Program} and 
{\tt Arduino with Tricolor LED and Push button}.

\begin{figure}
  \centering
  \includegraphics[width=\textwidth]{\LocLDRfig/LDR.png}
  \caption{An LDR to read its values with \arduino\ using a breadboard}
  %\redcolor{connected on pin no. D12}}
  \label{fig:ard-ldr}
\end{figure}

In case you have an LDR, and you want to connect it with \arduino\ on a breadboard, 
please refer to \figref{fig:ard-ldr}. The connections given in this figure can be 
used to read the voltage values from an LDR connected to the analog pin 5 on 
\arduino\ board. As shown in \figref{fig:ard-ldr}, one leg of the LDR is connected 
to 5V on \arduino\ and the other leg to the analog pin 5 on  \arduino. A resistor is also connected to the same leg and grounded. 
From \figref{fig:ldrconn} and \figref{fig:ard-ldr}, one can infer that a resistor 
along with the LDR is used to create a voltage divider circuit. The varying 
resistance of the LDR is converted to a varying voltage. Finally, this voltage is used 
by the analog pin 5 of \arduino\ in its logic. 

The connections shown in \figref{fig:ard-ldr-led} can be used to control an RGB LED, 
depending on the voltage values from the LDR. As shown in \figref{fig:ard-ldr-led}, 
digital pin 11 on \arduino\ is connected to the leftmost leg of the RGB LED. Rest of the connections
are same as that in \figref{fig:ard-ldr}. 

\begin{figure}
  \centering
  \includegraphics[width=\textwidth]{\LocLDRfig/LDR-led.png}
  \caption{An LDR to control an LED with Arduino Uno using a breadboard}
  %\redcolor{connected on pin no. D12}}
  \label{fig:ard-ldr-led}
\end{figure}


\section{Interfacing the LDR through the Arduino IDE}
\subsection{Interfacing the LDR}
In this section, we shall describe an experiment that will help 
to read the voltage values from an LDR connected to the analog pin 5 
of the \arduino\ board. Later, the read values will be used to change the state of an LED.  The shield has to be attached to the \arduino\ board
before doing these experiments and the \arduino\ needs to be connected to the computer 
with a USB cable, as shown in \figref{arduino}. The reader should go through the
instructions given in \secref{sec:ard-start} before getting started.

\begin{enumerate}
\item A simple code to read the LDR values is given in
  \ardref{ard:ldr-read}. As discussed earlier, the 0-5V LDR readings
  are mapped to 0-1023 through an ADC. The 
%\redcolor{Arduino IDE}\ 
  Arduino IDE
  based command for the analog read functionality is given by,
  \lstinputlisting[firstline=6,lastline=6]
  {\LocLDRardcode/ldr-read/ldr-read.ino} where {\tt A5} represents the
  analog pin 5 to be read and the read LDR values are stored in the
  variable {\tt val}.  The read values are then displayed using,
  \lstinputlisting[firstline=7,lastline=7]
  {\LocLDRardcode/ldr-read/ldr-read.ino} The delay in the code  
  \lstinputlisting[firstline=8,lastline=8]
  {\LocLDRardcode/ldr-read/ldr-read.ino} is added so that the readings
  do not scroll away very fast.  The entire reading and display
  operation is carried out 20 times. 

  To observe the values, one has to open the {\tt Serial Monitor} of
  the Arduino IDE.  The numbers displayed are in the range 0 to 1023
  and depend on the light falling on the LDR.  If one does the
  experiment in a completely dark room, the reading will be 0.  If on
  the other hand, a bright light, say for instance the torch light
  from mobile, is shined, the value displayed is close to 1023.  One
  will get intermediate values by keeping one's finger on the LDR. 
  While running this experiment, the readers must keep their fingertips on the LDR and
  observe the change in values being printed on the
  {\tt Serial Monitor} of Arduino IDE.

\item This experiment is an extension of the previous
experiment. Here, depending the resistance of the LDR, we will
  turn the red LED on.  The program for this is available at
  \ardref{ard:ldr-led}.  The value of LDR is read and stored in {\tt
    val}.  In case it is below some threshold (like 300 in \ardref{ard:ldr-led}), 
    it puts a high in pin number 11.  From \secref{sec:led-pril}, 
    one can note that this pin is for the red LED.  If the LDR value is below 300, 
    the red LED will be on, else, it will be turned off.  
    This loop is repeated 2,000 times. While running this experiment, the readers 
    must keep their fingertips on the LDR so that the threshold is achieved. Accordingly, 
    they can observe whether the red LED is turned on. 
\end{enumerate}

\begin{exercise}
Carry out the following exercise:
\begin{enumerate}
\item Carry out the experiment in a dark room and check what values
  get displayed on the {\tt Serial Monitor}.
\item Carry out the experiment with the torch light from the mobile
  phone shining on the LDR.
\end{enumerate}
\end{exercise}

\subsection{Arduino Code}
\label{sec:ldr-arduino-code}
\addtocontents{ard}{\protect\addvspace{\codclr}}

\begin{ardcode}
\acaption{Read and display the LDR values}
{Read and display the LDR values.  Available at
  \LocLDRardbrief{ldr-read/ldr-read.ino}.}
\label{ard:ldr-read}
\lstinputlisting{\LocLDRardcode/ldr-read/ldr-read.ino}
\end{ardcode}

\begin{ardcode}
\acaption{Turning the red LED on and off}
{Turning the red LED on and off.  Available at
  \LocLDRardbrief{ldr-led/ldr-led.ino}.}
\label{ard:ldr-led}
\lstinputlisting{\LocLDRardcode/ldr-led/ldr-led.ino}
\end{ardcode}

\section{Interfacing the LDR through Scilab}
\subsection{Interfacing the LDR}
In this section, we discuss how to carry out the experiments of the
previous section from Scilab. We will list the same two experiments,
in the same order.  The shield has to be attached to the \arduino\ board
before doing these experiments and the \arduino\ needs to be connected to the computer 
with a USB cable, as shown in \figref{arduino}.
The reader should go through the instructions given in
\secref{sec:sci-start} before getting started. 

% In this section, we will explain a few Scilab experiments to read the
% LDR values corresponding to the incident light. The LDR values can be
% read using the following function of Scilab Arduino toolbox:
% \begin{lstlisting}[style=nonumbers]
%   cmd_analog_in(1,port number on Arduino Uno)
% \end{lstlisting}
% where the input argument 1 is fixed for this kit, and the port number corresponds to the analog pin of \arduino that needs to be read.  We will carry out two experiments using Scilab.

\begin{enumerate}
\item In the first experiment, we will read the LDR values and display it in 
\scilab\ Console. The code for this experiment is 
given in  \sciref{sci:ldr-read}. As explained earlier in \secref{sec:light-sci}, 
we begin with serial port initialization. Then, we read the input coming from
analog pin 5 using the following command:  
\lstinputlisting[firstline=4,lastline=4]
  {\LocLDRscicode/ldr-read.sce}
  Note that the one leg of the LDR on
  the shield is connected to analog pin 5 of \arduino\, 
  as given in \figref{fig:ldrconn}. The read value is displayed in the 
  \scilab\ Console by the following command: 
  \lstinputlisting[firstline=5,lastline=5]
  {\LocLDRscicode/ldr-read.sce} where {\tt val} contains
  the LDR values ranging from 0 to 1023. If one does the experiment in a completely dark room, the
  reading will be 0. If on the other hand, a bright light, say for instance the torch
  light from mobile, is shined, the value displayed is close to 1023. One will get
  intermediate values by keeping one’s finger on the LDR. To
  encourage the user to have a good hands-on, we run these commands in
  a {\tt for} loop for 500 iterations. While running this experiment, the readers must keep their fingertips on the LDR and
  observe the change in values being printed on the \scilab\ Console. 


% We use \sciref{sci:ldr-read} to read the LDR values.  We find the
%   port number from the computer settings and give it as input to the
%   {\tt open\_serial} command to start serial port communication. In
%   our case, the port number is 2. Next, we shall fetch LDR values
%   using the command, {\tt cmd\_analog\_in}, as explained above. This
%   is indicated on line 4 of the code. We run this command in a {\tt
%     for} loop 20 times. In each iteration of the {\tt for} loop, we
%   acquire LDR data fed to analog pin 5, display it in the Scilab
%   command window and suspend Scilab operation for 500
%   milliseconds. The output of this experiment is displayed on the Scilab command
%   window. After reading the values, we close the serial port using the
%   command, {\tt close\_serial}, of Scilab-Arduino toolbox.

% \item In this experiment, we will observe the saturation point of LDR,
%   see \sciref{sci:ldr-led}.  We know that as incident light intensity
%   increases, voltage at analog input of the \arduino\ board
%   increases. Thus the ADC values being read by the \arduino\ board also
%   increase. But after certain high intensity, ADC values reach its
%   maximum. For 10 bit ADC in Arduino, this high intensity corresponds
%   to 1023.  Beyond this value, the LDR is incapable of sensing the
%   change in light intensity and is considered to be saturated. To
%   observe this saturation point, we can do a simple task of exposing
%   LDR to high intensity. We can put a torch/light source sensor to
%   close proximity of LDR.
\item This experiment is an extension of the previous
experiment. Here, depending the resistance of the LDR, we will
  turn the red LED on.  The program for this is available at
  \sciref{sci:ldr-led}.  The value of LDR is read and stored in {\tt
    val}.  In case it is below some threshold (like 300 in \ardref{ard:ldr-led}), 
    it puts a high in pin number 11.  From \secref{sec:led-pril}, 
    one can note that this pin is for the red LED.  If the LDR value is below 300, 
    the red LED will be on, else, it will be turned off.  
    This loop is repeated 2,000 times. While running this experiment, the readers 
    must keep their fingertips on the LDR so that the threshold is achieved. Accordingly, 
    they can observe whether the red LED is turned on. 
\end{enumerate}

\begin{exercise}
Carry out the exercise below:
\begin{enumerate}
\item Carry out the exercise in the previous section. 
\item Calculate the difference in LDR readings in indoor room
  before lighting the lamp and after lighting the lamp. You can also
  record changes in the room lighting at different times of the day.
\end{enumerate}
\end{exercise}

\subsection{Scilab Code}
\label{sec:ldr-scilab-code}
\addtocontents{cod}{\protect\addvspace{\codclr}}

\begin{scicode}
\ccaption{Read and display the LDR values}
{Read and display the LDR values.  Available at
  \LocLDRscibrief{ldr-read.sce}.}
\label{sci:ldr-read}
\lstinputlisting{\LocLDRscicode/ldr-read.sce}
\end{scicode}

\begin{scicode}
\ccaption{Turning the red LED on and off}
{Turning the red LED on and off.  Available at
  \LocLDRscibrief{ldr-led.sce}.}
\label{sci:ldr-led}
\lstinputlisting{\LocLDRscicode/ldr-led.sce}
\end{scicode}

\section{Interfacing the LDR through Xcos}
Next, we shall perform the above mentioned experiment, to read LDR
values, through Xcos.  We will carry out the same two experiments as in the previous
sections.  For each, will give the location
of the zcos file and the parameters to set.  The reader should go
through the instructions given in \secref{sec:xcos-start} before
getting started.

\begin{enumerate}
\item First we will read the LDR values and display it.  When the
file required for this experiment is invoked, one gets the GUI as in
\figref{fig:ldr-read}.  In the caption of this figure, one
can see where to locate the file. 

As discussed in earlier chapters, we start with the initialization
of the serial port. Next, using {\tt Analog Read} block, we read
the values of LDR connected on analog pin 5. Next, we use a scope to plot the values 
coming from this pin. When this Xcos file is simulated, a plot is opened, as shown in 
Fig. 

  \begin{figure}
    \centering
    \includegraphics[width=\smfig]{\LocLDRfig/ldr-read-xcos.PNG}
    \caption[Xcos diagram to read LDR values]{Xcos diagram to read LDR
      values.  
      This is what one sees when 
      \LocLDRscibrief{ldr-read.zcos}, is invoked.}
    \label{fig:ldr-read}
  \end{figure}

  We will next explain how to set the parameters for this simulation.
  To set value on any block, one needs to right click and open the
  {\tt Block Parameters} or double click.  The values for each block
  is tabulated in \tabref{tab:ldr-read}.  All other parameters are to
  be left unchanged.
  \begin{table}
    \centering
    \caption{Xcos parameters to read LDR}
    \label{tab:ldr-read}
    \begin{tabular}{llc} \hline
      Name of the block & Parameter name & Value \\ \hline
      ARDUINO\_SETUP & Identifier of Arduino Card & 1 \\
      & Serial com port number & 2\portcmd \\ \hline
      TIME\_SAMPLE & Duration of acquisition(s) & 10 \\
      & Sampling period(s) & 0.1 \\ \hline
      ANALOG\_READ\_SB & Analog Pin & 5 \\
      & Arduino card number & 1 \\ \hline
      CSCOPE & Ymin & 0 \\ 
      & Ymax & 1023 \\
      & Refresh period & 100 \\ \hline
      CLOCK\_c & Period & 0.1 \\
      & Initialisation Time & 0 \\ \hline
    \end{tabular}
  \end{table}

  During this experiment, we vary the light incident on LDR by using
   light sources and obstacles such as torch light, paper,
  hand (or fingertips), etc. and observe the LDR readings in the plot, as shown in 
  Fig. We observe that with a constant light source, the LDR output saturates after some time. 
%The output for this experiment is shown in \figref{fig:ldrsatout}.

%   \begin{figure}
%     \centering
%     \includegraphics[width=\lgfig]{\LocLDRfig/ldr-sat-out.png}
%     \caption[LDR output for varying intensity of incident light, as
%     seen in Xcos] {LDR output for varying intensity of
%       incident light, as seen in Xcos.  This is what one sees when
%       {\tt \LocLDRscibrief/ldr-read-xcos.zcos} is invoked.}
%     \label{fig:ldrsatout}
%   \end{figure}

\item In the second experiment, we take a step further and control the
state of red LED in accordance with the LDR values. When the file required for this
  experiment is invoked, one gets the GUI as in \figref{fig:ldr-led}.
  In the caption of this figure, one can see where to locate the file.

  \begin{figure}
    \centering
    \includegraphics[width=\lgfig]{\LocLDRfig/ldr-led.png}
%    \includegraphics[width=\smfig]{\LocLDRfig/ldr-led-xcos.PNG}
    \caption[Xcos diagram to read the value of the LDR, which is used
    to turn the blue LED on or off] {Xcos diagram to read the value of
      the LDR, which is used to turn the blue LED on or off.  This is
      what one sees when \LocLDRscibrief{ldr-led-xcos.zcos}, is
      invoked.}
    \label{fig:ldr-led}
  \end{figure}

  We will next explain how to set the parameters for this simulation.
  To set value on any block, one needs to right click and open the
  {\tt Block Parameters} or double click.  The values for each block
  is tabulated in \tabref{tab:ldr-led}.  In the CSCOPE\_c block, the
  two values correspond to two graphs, one for digital write and other
  for analog read values.  All other parameters are to be left
  unchanged.
  \begin{table}
    \centering
    \caption{Xcos parameters to read LDR and regulate blue LED}
    \label{tab:ldr-led}
    \begin{tabular}{llc} \hline
      Name of the block & Parameter name & Value \\ \hline
      ARDUINO\_SETUP & Identifier of Arduino Card & 1 \\
      & Serial com port number & 2\portcmd \\ \hline
      TIME\_SAMPLE & Duration of acquisition(s) & 10 \\
      & Sampling period(s) & 0.1 \\ \hline
      ANALOG\_READ\_SB & Analog pin & 5 \\
      & Arduino card number & 1 \\ \hline
      CMSCOPE & Ymin & 0 0 \\ 
      & Ymax & 1 1023 \\
      & Refresh period & 100 100 \\ \hline
      CLOCK\_c & Period & 0.1 \\
      & Initialisation time & 0 \\ \hline
      SWITCH2\_m & Datatype & 1 \\
      & threshold & 300 \\
      & pass first input if field & 0 \\
      & use zero crossing & 1 \\ \hline
      DIGITAL\_WRITE\_SB & Digital pin & 9 \\
      & Arduino card number & 1 \\ \hline
    \end{tabular}
  \end{table}
\end{enumerate}

\section{Interfacing the LDR through Python}
\subsection{Interfacing the LDR}
In this section, we discuss how to carry out the experiments of the
previous section from Python.  We will list the same two experiments,
in the same order.  The shield has to be attached to the \arduino\ board
before doing these experiments and the \arduino\ needs to be connected to the computer 
with a USB cable, as shown in \figref{arduino}.
The reader should go through the instructions given in
\secref{sec:python-start} before getting started.

\begin{enumerate}
\item In the first experiment, we will read the LDR values. The code for this experiment is given in
\pyref{py:ldr-read}. As explained earlier in \secref{sec:light-py}, we begin with 
importing necessary modules followed by setting up the serial port. 
Then, we read the input coming from analog pin 5 using the
following command:
\lstinputlisting[firstline=26,lastline=26]
{\LocLDRpycode/ldr-read.py} Note that the one leg of the LDR on
the shield is connected to analog pin 5 of \arduino\, 
as given in \figref{fig:ldrconn}. The read value is displayed 
 by the following command: 
\lstinputlisting[firstline=27,lastline=27]
{\LocLDRpycode/ldr-read.py} where {\tt val} contains
the LDR values ranging from 0 to 1023. If one does the experiment in a completely dark room, the
reading will be 0. If on the other hand, a bright light, say for instance the torch
light from mobile, is shined, the value displayed is close to 1023. One will get
intermediate values by keeping one's finger on the LDR. To
encourage the user to have a good hands-on, we run these commands in
a {\tt for} loop for 20 iterations. While running this experiment, the readers must keep their fingertips on the LDR and
observe the change in values being printed on on the
Command Prompt (on Windows) or Terminal (on Linux), as the case maybe.

\item This experiment is an extension of the previous experiment. Here, depending the resistance of the LDR, we will
  turn the red LED on.  The program for this is available at
  \pyref{py:ldr-led}.  The value of LDR is read and stored in {\tt
    val}.  In case it is below some threshold (like 300 in \pyref{py:ldr-led}), 
    it puts a high in pin number 11.  From \secref{sec:led-pril}, 
    one can note that this pin is for the red LED.  If the LDR value is below 300, 
    the red LED will be on, else, it will be turned off.  
    This loop is repeated 100 times. While running this experiment, the readers 
    must keep their fingertips on the LDR so that the threshold is achieved. Accordingly, 
    they can observe whether the red LED is turned on. 


% In this experiment, we will observe the saturation point of LDR,
%   see \pyref{py:ldr-led}.  We know that as incident light intensity
%   increases, voltage at analog input of the \arduino\ board
%   increases. Thus the ADC values being read by the \arduino\ board also
%   increase. But after certain high intensity, ADC values reach its
%   maximum. For 10 bit ADC in Arduino, this high intensity corresponds
%   to 1023.  Beyond this value, the LDR is incapable of sensing the
%   change in light intensity and is considered to be saturated. To
%   observe this saturation point, we can do a simple task of exposing
%   LDR to high intensity. We can put a torch/light source sensor to
%   close proximity of LDR.
\end{enumerate}

\begin{exercise}
Carry out the exercise below:
\begin{enumerate}
\item Carry out the exercise in the previous section. 
\item Calculate the difference in LDR readings in indoor room
  before lighting the lamp and after lighting the lamp. You can also
  record changes in the room lighting at different times of the day.
\end{enumerate}
\end{exercise}

\subsection{Python Code}
\label{sec:ldr-python-code}
\addtocontents{pyd}{\protect\addvspace{\codclr}}

\begin{pycode}
\pcaption{Read and display the LDR values}
{Read and display the LDR values.  Available at
  \LocLDRpybrief{ldr-read.py}.}
\label{py:ldr-read}
\lstinputlisting{\LocLDRpycode/ldr-read.py}
\end{pycode}

\begin{pycode}
\pcaption{Turning the red LED on and off}
{Turning the red LED on and off.  Available at
  \LocLDRpybrief{ldr-led.py}.}
\label{py:ldr-led}
\lstinputlisting{\LocLDRpycode/ldr-led.py}
\end{pycode}

\section{Interfacing the LDR through Julia}
\subsection{Interfacing the LDR}
In this section, we discuss how to carry out the experiments of the
previous section from Julia.  We will list the same two experiments,
in the same order.  The shield has to be attached to the \arduino\ board
before doing these experiments and the \arduino\ needs to be connected to the computer 
with a USB cable, as shown in \figref{arduino}.
The reader should go through the instructions given in \secref{sec:julia-start} before getting started.


\begin{enumerate}
\item In the first experiment, we will read the LDR values. The code for this experiment is given in
\juliaref{julia:ldr-read}. As explained earlier in \secref{sec:light-julia}, we begin with importing the SerialPorts 
\cite{julia-serial-ports} package and the module ArduinoTools followed by setting up the serial port.
Then, we read the input coming from analog pin 5 using the 
following command:
\lstinputlisting[firstline=6,lastline=6]
{\LocLDRjuliacode/ldr-read.jl} Note that the one leg of the LDR on
the shield is connected to analog pin 5 of \arduino\, 
as given in \figref{fig:ldrconn}. The read value is displayed 
 by the following command: 
\lstinputlisting[firstline=7,lastline=7]
{\LocLDRjuliacode/ldr-read.jl} where {\tt val} contains
the LDR values ranging from 0 to 1023. If one does the experiment in a completely dark room, the
reading will be 0. If on the other hand, a bright light, say for instance the torch
light from mobile, is shined, the value displayed is close to 1023. One will get
intermediate values by keeping one's finger on the LDR. To
encourage the user to have a good hands-on, we run these commands in
a {\tt for} loop for 20 iterations. While running this experiment, the readers must keep their fingertips on the LDR and
observe the change in values being printed on on the
Command Prompt (on Windows) or Terminal (on Linux), as the case maybe.


\item This experiment is an extension of the previous experiment. Here, depending the resistance of the LDR, we will
  turn the red LED on.  The program for this is available at
  \juliaref{julia:ldr-led}.  The value of LDR is read and stored in {\tt
    val}.  In case it is below some threshold (like 300 in \juliaref{julia:ldr-led}), 
    it puts a high in pin number 11. From \secref{sec:led-pril}, 
    one can note that this pin is for the red LED.  If the LDR value is below 300, 
    the red LED will be on, else, it will be turned off.  
    This loop is repeated 500 times. While running this experiment, the readers 
    must keep their fingertips on the LDR so that the threshold is achieved. Accordingly, 
    they can observe whether the red LED is turned on. 
\end{enumerate}

\begin{exercise}
Carry out the exercise below:
\begin{enumerate}
\item Carry out the exercise in the previous section. 
\item Calculate the difference in LDR readings in indoor room
  before lighting the lamp and after lighting the lamp. You can also
  record changes in the room lighting at different times of the day.
\end{enumerate}
\end{exercise}

\subsection{Julia Code}
\label{sec:ldr-julia-code}
\addtocontents{juliad}{\protect\addvspace{\codclr}}

\begin{juliacode}
\jcaption{Read and display the LDR values}
{Read and display the LDR values.  Available at
  \LocLDRjuliabrief{ldr-read.jl}.}
\label{julia:ldr-read}
\lstinputlisting{\LocLDRjuliacode/ldr-read.jl}
\end{juliacode}

\begin{juliacode}
\jcaption{Turning the red LED on and off}
{Turning the red LED on and off.  Available at
  \LocLDRjuliabrief{ldr-led.jl}.}
\label{julia:ldr-led}
\lstinputlisting{\LocLDRjuliacode/ldr-led.jl}
\end{juliacode}

\section{Interfacing the LDR through OpenModelica}
\subsection{Interfacing the LDR}
In this section, we discuss how to carry out the experiments of the
previous section from OpenModelica.  We will list the same two experiments,
in the same order.  The shield has to be attached to the \arduino\ board
before doing these experiments and the \arduino\ needs to be connected to the computer 
with a USB cable, as shown in \figref{arduino}.
The reader should go through the instructions given in
\secref{sec:OpenModelica-start} before getting started.

\begin{enumerate}
\item In the first experiment, we will read the LDR values. The code for this experiment is given in
\OpenModelicaref{OpenModelica:ldr-read} . As explained earlier in \secref{sec:light-OpenModelica}, 
  we begin with importing the two packages: Streams and SerialCommunication followed 
  by setting up the serial port. Then, we read the input coming from analog pin 5 using the 
following command:
\lstinputlisting[firstline=16,lastline=16]
{\LocLDROpenModelicacode/ldr-read.mo} Note that the one leg of the LDR on
the shield is connected to analog pin 5 of \arduino\, 
as given in \figref{fig:ldrconn}. The read value is displayed 
 by the following command: 
\lstinputlisting[firstline=17,lastline=17]
{\LocLDROpenModelicacode/ldr-read.mo} where {\tt val} contains
the LDR values ranging from 0 to 1023. If one does the experiment in a completely dark room, the
reading will be 0. If on the other hand, a bright light, say for instance the torch
light from mobile, is shined, the value displayed is close to 1023. One will get
intermediate values by keeping one's finger on the LDR. While simulating this experiment, the readers must keep their fingertips on the LDR and
observe the change in values being printed on on the output window of OMEdit, as shown in \figref{om-sim-success}.

\item This experiment is an extension of the previous experiment. Here, depending the resistance of the LDR, we will
  turn the red LED on.  The program for this is available at
  \OpenModelicaref{OpenModelica:ldr-led}.  The value of LDR is read and stored in {\tt
    val}.  In case it is below some threshold (like 300 in \OpenModelicaref{OpenModelica:ldr-led}), 
    it puts a high in pin number 11. From \secref{sec:led-pril}, 
    one can note that this pin is for the red LED. If the LDR value is below 300, 
    the red LED will be on, else, it will be turned off. While running this experiment, the readers 
    must keep their fingertips on the LDR so that the threshold is achieved. Accordingly, 
    they can observe whether the red LED is turned on. 
\end{enumerate}

\subsection{OpenModelica Code}
\label{sec:ldr-OpenModelica-code}
\addtocontents{OpenModelicad}{\protect\addvspace{\codclr}}

\begin{OpenModelicacode}
\mcaption{Read and display the LDR values}
{Read and display the LDR values.  Available at
  \LocLDROpenModelicabrief{ldr-read.mo}.}
\label{OpenModelica:ldr-read}
\lstinputlisting{\LocLDROpenModelicacode/ldr-read.mo}
\end{OpenModelicacode}

\begin{OpenModelicacode}
\mcaption{Turning the red LED on and off}
{Turning the red LED on and off.  Available at
  \LocLDROpenModelicabrief{ldr-led.mo}.}
\label{OpenModelica:ldr-led}
\lstinputlisting{\LocLDROpenModelicacode/ldr-led.mo}
\end{OpenModelicacode}





%%%%%%%%OpenMocelica Description Ends 

% \section{Do we need these?  \redcolor{Manas, please answer}}
% \begin{figure}
% \centering
% \includegraphics[width=\lgfig]{\LocLDRfig/ldr-led.png}
% \caption{Xcos diagram to change LED state depending on the LDR values}
% \label{fig:ldrxcosled}
% \end{figure}

% \begin{figure}
% \centering
% \includegraphics[width=\lgfig]{\LocLDRfig/ldr-led-out.png}
% \caption{LDR output and corresponding LED input}
% \label{fig:ldrledout}
% \end{figure}



%%%%%%%%%%%%%%python code

%%%%%%%%%%%%%%python code

%%%%%begin julia code

%%%%%end julia


%%%%%begin OpenModelica code
%%%%%%%%%%%%%%%%%OpenModelica ends
