\chapter{Controlling a DC motor}
\thispagestyle{empty}
\label{dcmotor}
\newcommand{\LocDCMfig}{\Origin/user-code/dcmotor/figures}
\newcommand{\LocDCMscicode}{\Origin/user-code/dcmotor/scilab}
\newcommand{\LocDCMscibrief}[1]{{\tt \seqsplit{
        Origin/user-code/dcmotor/scilab/#1}},
  see \fnrefp{fn:file-loc}}
\newcommand{\LocDCMardcode}{\Origin/user-code/dcmotor/arduino}
\newcommand{\LocDCMardbrief}[1]{{\tt \seqsplit{
        Origin/user-code/dcmotor/arduino/#1}},
  see \fnrefp{fn:file-loc}}

%%%%%%%%%%%%%python starts
\newcommand{\LocDCMpycode}{\Origin/user-code/dcmotor/python}
\newcommand{\LocDCMpybrief}[1]{{\tt \seqsplit{
        Origin/user-code/dcmotor/python/#1}},
  see \fnrefp{fn:file-loc}}
%%%%%%%%%%%%%python ends

%%%%%%%%%%%%%julia starts
\newcommand{\LocDCMjuliacode}{\Origin/user-code/dcmotor/julia}
\newcommand{\LocDCMjuliabrief}[1]{{\tt \seqsplit{
        Origin/user-code/dcmotor/julia/#1}},
  see \fnrefp{fn:file-loc}}
%%%%%%%%%%%%%julia ends

%%%%%%OpenModelica Starts
\newcommand{\LocDCMOpenModelicacode}{\Origin/user-code/dcmotor/OpenModelica}  %added for OpenModelica
\newcommand{\LocDCMOpenModelicabrief}[1]{{\tt \seqsplit{%
        Origin/user-code/led/OpenModelica/#1}}, see \fnrefp{fn:file-loc}} % added for OpenModelica

%%%%%OpenModelcia Ends

Motors are widely used in commercial applications. 
DC motor converts electric power obtained from direct current to 
mechanical motion. This chapter describes the experiments to 
control DC motor with \arduino\ board. We will observe the 
direction of motion of the DC motor being changed 
using the microcontroller on \arduino\ board. 
Control instruction will be sent to \arduino\ using Arduino IDE, 
Scilab scripts, Scilab Xcos, Python, Julia, and OpenModelica. 
The experiments provided in this chapter don't require the shield. 
Therefore, the readers must remove the shield from the \arduino\ before 
moving further in this chapter. Before removing the shield, 
the readers are advised to detach \arduino\ from the computer. 

\section{Preliminaries}
In order to change its direction, the sign of the voltage applied to
the DC motor is changed.  For that, one needs to use external hardware
called \index{H-Bridge circuit DC motor}%
H-Bridge circuit DC motor with \arduino. \index{H-Bridge}%
H-Bridge allows direction of the current passing through the DC motor
to be changed. It avoids the sudden short that may happen while
changing the direction of current passing through the motor.  It is
one of the essential circuits for the smooth operation of a DC
motor. There are many manufacturers of H-bridge circuit viz.
\index{L293D,L298}%
L293D, L298, etc.  Often they provide small \index{PCB breakout
  board}%
PCB breakout boards.  These modules also provide an extra supply that
is needed to drive the DC motor.  \figref{fig:motordriverboard} shows
the diagram of a typical breakout board containing IC L293D, which will
be used in this book. \par

\begin{figure}
  \centering
  \includegraphics[width=\lgfig]{\LocDCMfig/dcmotor_board.png}
  \caption{L293D motor driver board}
  \label{fig:motordriverboard}
\end{figure}

Input from \arduino\ to H-bridge IC is in \index{pulse width
  modulation, PWM}%
pulse width modulation (PWM) form. PWM is a technique to generate
analog voltages using digital pins. We know that \arduino\ has digital
input-output pins. When these pins are configured as an output, they
provide High (5V) or Low (0V) voltage. With PWM technique, these pins
are switched on and off iteratively and fast enough so that the
voltage is averaged out to some analog value in between 0-5V. This
analog value depends on ''switch-on'' time and ''switch-off''
time. For example, if both ''switch-on'' time and ''switch-off'' time
are equal, average voltage on PWM pin will be 2.5V. To enable fast
switching of digital pin, a special hardware is provided in
microcontrollers.  PWM is considered as an important resource of
the microcontroller system. \arduino\ board has 6 PWM pins (3, 5, 6, 9, 10, 11) \cite{arduino-pwm}. 
On an original \arduino\ board, these pins are marked with a tilde sign next to the pin number, 
as shown in Fig. For each of these pins, the input can come from 8 bits.
Thus we can generate $2^8 = 256$ different analog values (from 0 to 255) 
in between 0-5V with these pins.

We now carry out the following connections:
\begin{enumerate}
  \item Connect input of L293D (M1\_IN) pins to two of the PWM pins
        available on \arduino.  We have used pins 9 and 10 of the \arduino\
        board. 
  \item Connect the output of the L293D (M1\_OUT) pins directly to the 2
        wires of the DC motor.  As the direction is changed during the
        operation, the polarity of the connection does not matter.
  \item Connect supply (Vcc) and ground (Gnd) pins of L293D to 5V and
        Gnd pins of the \arduino\ board, respectively.
\end{enumerate}
A schematic of these connections is given in
\figref{fig:dcm-schematic}.  The actual connections can be seen in
\figref{fig:dcmotorconn}.


\begin{figure}
  \centering
  \includegraphics[width=\smfig]{\LocDCMfig/schematic.png}
  \caption{A schematic of DC motor connections}
  \label{fig:dcm-schematic}
\end{figure}
\begin{figure}
  \centering
  \includegraphics[width=\lgfig]{\LocDCMfig/dc_motor_description.jpg}
  \caption{How to connect the DC motor to the \arduino\ board}
  \label{fig:dcmotorconn}
\end{figure}

\section{Controlling the DC motor from Arduino}
\subsection{Controlling the DC motor}
\label{sec:dcm-ard}
In this section, we will describe some experiments that will help
drive the DC motor from the Arduino IDE.  We will also give the
necessary code.  We will present three experiments in this section. 
As mentioned earlier, the shield must be removed from 
the \arduino\ and the \arduino\ needs to be connected to the computer 
with a USB cable, as shown in \figref{arduino}. The reader should go through the
instructions given in \secref{sec:ard-start} before getting started. 

\paragraph{Note:} The readers are advised to affix a small 
(very lightweight) piece of paper at the tip of the shaft of the DC motor. 
That will help them observe the direction of rotation 
of the DC motor while running the experiments. 

\begin{enumerate}
  \item We now demonstrate how to drive the DC motor from the Arduino
        IDE.  \ardref{ard:dcmotor-clock} has the required code for this.  It
        starts the serial port at a baud rate of 115200.  Pins 9 and 10 are
        declared as output pins and hence values can be written on to them. The following 
        lines are used to declare these two pins as output pins: 
        \lstinputlisting[firstline=3,lastline=4]
        {\LocDCMardcode/dcmotor-clock/dcmotor-clock.ino}
        Next, we write PWM 100 on pin 9 and PWM 0 on pin 10, as shown below:
        \lstinputlisting[firstline=5,lastline=6]
        {\LocDCMardcode/dcmotor-clock/dcmotor-clock.ino}
        Recall from \figref{fig:dcmotorconn} that pins 9 and 10 are connected to the
        input of the breakout board, which in turn makes the DC motor run at
        an intermediate speed.  Some of the breakout boards may not have enough current driving
        capability and hence tend to heat up.  To avoid these difficulties,
        the DC motor is run at an intermediate value of PWM 100. Remember, we can 
        increase this value upto 255. 
        % As mentioned in \secref{sec:led-pril}, a high on pin 9 also makes the blue LED turn on. 
        
        The line containing {\tt delay} makes the previous command execute
        for 3 seconds.  As a result, the DC motor continues to rotate for 3
        seconds.  After this, we put a 0 in both pins 9 and 10, as shown below:
        \lstinputlisting[firstline=8,lastline=9]
        {\LocDCMardcode/dcmotor-clock/dcmotor-clock.ino}
        With this, the motor comes to a halt.  
        % The blue LED is also turned off.
        
  \item It is easy to make the DC motor run in the reverse direction by
        interchanging the values put on pins 9 and 10.  This is done in
        \ardref{ard:dcmotor-both}.  In this code, we make the DC motor
        run in one direction for 3 seconds and then make it rotate in the
        reverse direction for 2 seconds.  The rotation in reverse direction
        is achieved by putting 100 in pin 10, as shown below:
        \lstinputlisting[firstline=8,lastline=9]
        {\LocDCMardcode/dcmotor-both/dcmotor-both.ino}
        % This makes the green LED light up as well, recall the discussion in \secref{sec:led-pril}.  After
        Next, we release the motor by writing 0 in both pins 9 and 10, as shown below:
        \lstinputlisting[firstline=11,lastline=12]
        {\LocDCMardcode/dcmotor-both/dcmotor-both.ino}
        With this, the motor comes to a halt.  
        % This turns the green LED off as well.
        
  \item Next, we make the DC motor run in forward and reverse
        directions, in a loop.  This is done through
        \ardref{ard:dcmotor-loop}.  We first write PWM 100 in pin 9 for 3
        seconds.  After that, make the motor stop for 2 seconds.  Finally,
        make the motor rotate in the reverse direction by writing PWM 100 in pin 10
        for two seconds.  Finally, we make the motor stop for one second.
        The entire thing is put in a {\tt for} loop which runs for 5 iterations. 
\end{enumerate}

\begin{exercise}
  Carry out the following exercise:
  \begin{enumerate}
    \item Try out some of the suggestions given above, \ie\ removing
          certain numbers from the code
    \item See if the DC motor runs if you put 1 instead of 100 as the PWM
          value.  Explain why it does not run.  Find out the smallest value at
          which it will start running.
  \end{enumerate}
\end{exercise}

\subsection{Arduino Code}
\label{sec:dcmotor-arduino-code}
\lstset{style=mystyle}
\addtocontents{ard}{\protect\addvspace{\codclr}}

\begin{ardcode}
  \acaption{Rotating the DC motor}
  {Rotating the DC motor.  Available at
    \LocDCMardbrief{dcmotor-clock/dcmotor-clock.ino}.}
  \label{ard:dcmotor-clock}
  \lstinputlisting{\LocDCMardcode/dcmotor-clock/dcmotor-clock.ino}
\end{ardcode}

\begin{ardcode}
  \acaption{Rotating the DC motor in both directions}{Rotating the DC
    motor in both directions.  
    Available at
    \LocDCMardbrief{dcmotor-both/dcmotor-both.ino}.}
  \label{ard:dcmotor-both}
  \lstinputlisting{\LocDCMardcode/dcmotor-both/dcmotor-both.ino}
\end{ardcode}

\begin{ardcode}
  \acaption{Rotating the DC motor in both directions in a loop}{Rotating
    the DC motor in both directions in a loop.
    Available at
    \LocDCMardbrief{dcmotor-loop/dcmotor-loop.ino}.}
  \label{ard:dcmotor-loop}
  \lstinputlisting{\LocDCMardcode/dcmotor-loop/dcmotor-loop.ino}
\end{ardcode}


\section{Controlling the DC motor from Scilab}
\label{sec:dcm-sci}
\subsection{Controlling the DC motor}
In this section, we discuss how to carry out the experiments of the
previous section from Scilab. We will list the same three experiments,
in the same order.  As mentioned earlier, the shield must be removed from 
the \arduino\ and the \arduino\ needs to be connected to the computer 
with a USB cable, as shown in \figref{arduino}. The reader should go through the instructions given in
\secref{sec:sci-start} before getting started. 

\paragraph{Note:} The readers are advised to affix a small 
(very lightweight) piece of paper at the tip of the shaft of the DC motor. 
That will help them observe the direction of rotation 
of the DC motor while running the experiments. 

\begin{enumerate}
  \item In the first experiment, we will learn how to drive the DC motor
        from Scilab. The code for this experiment is 
        given in  \sciref{sci:dcmotor-clock}. As explained earlier in \secref{sec:light-sci}, 
        we begin with serial port initialization. 
        Next, the code has a command of the following form: 
        \begin{lstlisting}[style=nonumbers]
              cmd_dcmotor_setup(1, H-Bridge type, Motor number, PWM pin 1, PWM pin 2)
        \end{lstlisting}
        As mentioned earlier, this chapter makes use of an H-Bridge circuit which 
        allows direction of the current passing through the DC motor to be changed.
        We are using L293D as an H-Bridge circuit in this book. Thus, we will pass the value 3 for
        H-Bridge type. The Scilab-Arduino toolbox, as explained in \secref{sec:sci-ard-toolbox}, 
        supports three types of H-Bridge circuit. \tabref{table:convention}
        provides the values to be passed for different H-Bridge circuits. 
        Next argument in the command given above is Motor number. Here, we pass the value 1. 
        Finally, we provide the PWM pins to which the DC motor is connected. As 
        shown in \figref{fig:dcmotorconn}, pins 9 and 10 are connected to the
        input of the breakout board. As a result, the command {\tt cmd\_dcmotor\_setup} becomes
        \lstinputlisting[firstline=2,lastline=2]
        {\LocDCMscicode/dcmotor-clock.sce}
        
        \begin{table}
          \centering
          \caption{Values in the \scilab\ command for different H-Bridge circuits}
          \label{table:convention}
          \begin{tabular}{llc}\hline
            Type of H-Bridge circuit & Value \\ \hline
            MotorShield Rev3         & 1                             \\ \hline 
            PMODHB5/L298             & 2                             \\ \hline 
            L293D                    & 3                             \\ \hline
          \end{tabular}
        \end{table}
        
        The next line of \sciref{sci:dcmotor-clock} is of the following form: 
        \begin{lstlisting}[style=nonumbers]
          cmd_dcmotor_run(1, Motor number, [sign] PWM value)
        \end{lstlisting}
        Here, we will pass the value 1 in Motor number.  As mentioned earlier, 
        for each of the PWM pins on \arduino\ board, the input can come from 8 bits.
        Thus, these pins can supply values between $- 255$ and $+ 255$. Positive values correspond to clockwise
        rotation while negative values correspond to anti-clockwise rotation. Based on the PWM value and polarity, 
        corresponding analog voltage is generated.  
        We put a PWM value of 100 to make the DC motor run at an intermediate speed.  
        As a result, the command {\tt cmd\_dcmotor\_run} becomes
        \lstinputlisting[firstline=3,lastline=3]
        {\LocDCMscicode/dcmotor-clock.sce}
        
        The above-mentioned command does not say for how long the motor should run.  This is taken care of
        by the {\tt sleep} command, as given below:
        \lstinputlisting[firstline=4,lastline=4]{\LocDCMscicode/dcmotor-clock.sce}
        With this, the DC motor will run for 3000 milliseconds or 3 seconds. At last, 
        we release the DC motor, as shown below:
        \lstinputlisting[firstline=5,lastline=5]{\LocDCMscicode/dcmotor-clock.sce}
        With the execution of this command, the PWM functionality on the \arduino\ pins
        is ceased.  This has the motor number as an input
        parameter. At last, we close the serial port. 
        
        
        \paragraph{Note:} If the sleep command (at line 4 of \sciref{sci:dcmotor-clock}) 
        were not present, the DC motor will not even run: soon after putting the value 100, 
        the DC motor would be released, leaving no time in between.  On the other hand, if
        the DC motor is not released (\ie\ line number 5 of \sciref{sci:dcmotor-clock} being commented), 
        the DC motor will go on rotating. That's why, it may be inferred that 
        line number 5 of \sciref{sci:dcmotor-clock} is mandatory
        for every program. We encourage the readers to run  \sciref{sci:dcmotor-clock} by commenting
        any one or two of the lines numbered 4, 5 or 6.  Go ahead and do it - you will not break
        anything.  At the most, you may have to unplug the USB cable connected to \arduino\ and
        restart the whole thing from the beginning.
        
  \item It is easy to make the DC motor run in the reverse direction by
        changing the sign of PWM value being written.  This is done in
        \sciref{sci:dcmotor-both}.  In this code, we make the DC motor
        run in one direction for 3 seconds and then make it rotate in the
        reverse direction for 2 seconds.  The rotation in reverse direction
        is achieved by putting $- 100$ in the command {\tt cmd\_dcmotor\_run}, 
        as shown below:
        \lstinputlisting[firstline=5,lastline=5]
        {\LocDCMscicode/dcmotor-both.sce}
        % This makes the green LED light up as well, recall the discussion in \secref{sec:led-pril}.  After
        After adding a {\tt sleep} of 2 seconds, we release the motor by issuing
        the command {\tt cmd\_dcmotor\_release}, followed by closing the serial port:
        \lstinputlisting[firstline=7,lastline=8]
        {\LocDCMscicode/dcmotor-both.sce}
        With this, the motor comes to a halt.  
        % This turns the green LED off as well.
        
  \item Next, we make the DC motor run in forward and reverse
        directions, in a loop.  This is done through
        \sciref{sci:dcmotor-loop}.  We first write PWM $+100$ for 3
        seconds.  After that, halt the motor for 2 seconds by writing zero PWM value.  
        Next, make the motor rotate in the reverse direction by writing PWM $-100$ for two seconds.  
        Next, we make the motor stop for one second. This procedure is put in a {\tt for} loop which runs for 4 iterations.
        At last, we release the motor by issuing the command {\tt cmd\_dcmotor\_release}, followed by closing the serial port
        
\end{enumerate}

% \subsection{Initialization}
% In all the experiments in this section, we need to initialize the DC
% motor first, using a \scilab\ command of the following type:

% \begin{lstlisting}[style=nonumbers]
%   cmd_dcmotor_setup(1,H-Bridge type,Motor number,PWM pin 1,PWM pin 2)
% \end{lstlisting}
% As mentioned earlier, number 1 in the above list refers to the
% \arduino\ board.  We now discuss how to choose values for the other
% parameters in this command.  As mentioned above, there are many
% H-bridge IC manufacturers.  The inbuilt function {\tt
%     cmd\_dcmotor\_setup} can work with most of the widely used ICs,
% through a suitable input parameter.  Users have to provide the type
% number of the breakout board they have.  Popular numbering convention
% for different types of DC motor breakout boards is given in
% \tabref{table:convention}.  For example, L293D is type 3.  Next, we
% have to provide the motor number we want to control.  In our case, it
% is number 1.  Finally we want to provide PWM pin numbers on \arduino.
% As mentioned earlier, we are using pins 10 and 11.  In
% \tabref{tab:dcmotor-init}, we list the choices that we have made.
% Inserting these parameter values in the above shown \scilab\ command,
% we get the following command \\
% \lstinputlisting[firstline=2,lastline=2]
% {\LocDCMscicode/dcmotor-clock.sce}
% which is line number 2 in \sciref{sci:dcmotor-clock}.  We have already
% seen 
% the first two lines of this code and hence will not explain here.  We
% will add more lines to this code as we go along.




% % \begin{table}
% %   \centering
% %   \caption{Parameters for DC motor initialization}
% %   \label{tab:dcmotor-init}
% %   \begin{tabular}{|l|c|} \hline
% %     Parameter     & Value \\ \hline
% %     H-Bridge type & 3     \\ 
% %     Motor number  & 1     \\
% %     PWM 1 pin     & 9     \\
% %     PWM 2 pin     & 10    \\ \hline
% %   \end{tabular}
% % \end{table}

% \subsection{Rotation for a specified time}
% \label{sec:dc-both}
% We will now explain how to run the DC motor.  We have to provide motor
% number and the PWM value.  The \scilab\ command is of the form,
% \begin{lstlisting}[style=nonumbers]
%   cmd_dcmotor_run(1,Motor number,(sign)(PWM value))
% \end{lstlisting}
% Motor number is 1, as mentioned earlier.  Considering that the input
% to a PWM pin comes from two 8 digital pins, we can provide values
% between $-255$ and +255. Positive values correspond to clockwise
% rotation while negative values correspond to anti-clockwise rotation.
% Based on the PWM value and polarity, corresponding analog voltage is
% generated.  We put a PWM value of 100 to make the DC motor to
% run at an intermediate speed.  Assigning these values, we get the
% following command:
% \lstinputlisting[firstline=3,lastline=3]{\LocDCMscicode/dcmotor-clock.sce}
% This is line number 3 in \sciref{sci:dcmotor-clock}.  This command
% does not say for how long the motor should run.  This is taken care of
% by the {\tt sleep} statement.  The units of sleep are milliseconds.
% For example, line number 4 of \sciref{sci:dcmotor-clock}, given next,
% says that \scilab\ should go to sleep for three seconds.
% \lstinputlisting[firstline=4,lastline=4]{\LocDCMscicode/dcmotor-clock.sce}

% Line number 5 of \sciref{sci:dcmotor-clock}, shown below, is mandatory
% for every program.
% \lstinputlisting[firstline=5,lastline=5]{\LocDCMscicode/dcmotor-clock.sce}
% It releases the DC motor.  The PWM functionality on the \arduino\ pins
% is ceased using this command.  This has the motor number as an input
% parameter.

% If the sleep command discussed above were not present, the DC motor
% will not even run: soon after putting the value 100, the DC motor
% would be released, leaving no time in between.  If on the other hand,
% the DC motor is not released (\ie\ line number 6 being absent), the DC
% motor will go on rotating.  Line number 6 of \sciref{sci:dcmotor-clock}
% closes the serial port.

% We encourage you to run the above code without either line numbers 4,
% 5 or 6 or all combinations.  Go ahead and do it - you will not break
% anything.  At the most, you may have to unplug the USB cable and
% restart the whole thing from the beginning.

% \sciref{sci:dcmotor-clock} can easily be extended to make the DC motor
% run in both directions.  The modified code is available in
% \sciref{sci:dcmotor-both}.

\begin{exercise}
  Carry out the following exercise:
  \begin{enumerate}
    \item Try out some of the suggestions given above, \ie\ removing
          certain numbers from the code. 
    \item See if the DC motor runs if you put 1 instead of 100 as the PWM
          value.  Explain why it does not run.  Find out the smallest value at
          which it will start running.
  \end{enumerate}
\end{exercise}

% \subsection{Using the capabilities of \scilab}
% Given that Scilab has a powerful programming syntax, a lot of
% different experiments can be tried out.  We illustrate a few in this
% section.  We begin with a {\tt for loop}.

% In the previous section, we presented \sciref{sci:dcmotor-both}, where
% we made the motor run in both directions, five seconds in the
% clockwise direction and two seconds in reverse.  This code can be
% embedded in a loop and the motor be made to repeat a certain number of
% times.  This idea is implemented through \sciref{sci:dcmotor-loop}.
% Through the {\tt for loop} in between line numbers 3 and 8, we make
% the DC motor repeat four times the cycle containing one rotation in
% each direction. 
%  \figref{fig:dcmotorfc} explains the entire operation
% through a flowchart.
% \begin{figure}
% \centering
% \includegraphics[width=\lgfig]{\LocDCMfig/dcmotorflowchart.png}
% \caption{Flowchart}
% \label{fig:dcmotorfc}
% \end{figure}

% It is not difficult to see how some of the other features of the
% \scilab\ programming language can be used along with this DC motor.
% For example, it is possible to read a temperature value and based on
% its value, start or stop the motor.  For real world applications, one
% has to provide extra current carrying capabilities through external
% hardware.  

\subsection{Scilab Code}
\label{sec:dcmotor-scilab-code}
\addtocontents{cod}{\protect\addvspace{\codclr}}

\begin{scicode}
  \ccaption{Rotating the DC motor}
  {Rotating the DC motor.  Available at
    \LocDCMscibrief{dcmotor-clock.sce}.}
  \label{sci:dcmotor-clock}
  \lstinputlisting{\LocDCMscicode/dcmotor-clock.sce}
\end{scicode}

\begin{scicode}
  \ccaption{Rotating the DC motor in both directions}
  {Rotating DC motor in both directions.  Available at
    \LocDCMscibrief{dcmotor-both.sce}.}
  \label{sci:dcmotor-both}
  \lstinputlisting{\LocDCMscicode/dcmotor-both.sce}
\end{scicode}

\begin{scicode}
  \ccaption{Rotating the DC motor in both directions in a loop}{Rotating
    the DC motor in both directions in a loop.
    Available at
    \LocDCMscibrief{dcmotor-loop.sce}.}
  \label{sci:dcmotor-loop}
  \lstinputlisting{\LocDCMscicode/dcmotor-loop.sce}
\end{scicode}

\section{Controlling the DC Motor from Xcos}
In this section, we will see how to drive the DC motor from Scilab Xcos. 
We will carry out the same three experiments as in the previous
sections. For each experiment, we will give the location of the zcos file and the
parameters to set.  The reader should go through the instructions
given in \secref{sec:xcos-start} before getting started.  


% If the
% rotation of the DC motor is blocked by any obstacle in any of the
% experiments given below, you may want to hold it in your hand and let
% it run unhindered.

\begin{enumerate}
  \item First we will see a simple code that drives the DC motor for a
        specified time.  When the file required for this experiment is
        invoked, one gets the GUI as in \figref{fig:dcmotor-clock}.  In
        the caption of this figure, one can see where to locate the file.
        
        \begin{figure}
          \centering
          \includegraphics[width=\smfig]{\LocDCMfig/dcmotor-clock.png}
          \caption[Control of DC motor for a specified time from Xcos]
          {Control of DC motor for a specified time from Xcos.  This is what
            one sees when \LocDCMscibrief{dcmotor-clock.zcos}, is
            invoked.}
          \label{fig:dcmotor-clock}
        \end{figure}
        
        We will next explain how to set the parameters for this simulation.
        To set value on any block, one needs to right click and open the
          {\tt Block Parameters} or double click.  The values for each block
        is tabulated in \tabref{tab:dcmotor-clock}.  In case of {\tt
            DCMOTOR\_SB}, enter 3 to indicate for L293D board.  After clicking
        on OK, another dialog box will pop up.  In that, enter the PWM pin numbers
        as 9 and 10 and click OK.  
        All other parameters are to be left
        unchanged.
        \begin{table}
          \centering
          \caption{Xcos parameters to drive the DC motor for a specified time}
          \label{tab:dcmotor-clock}
          \begin{tabular}{llc} \hline
            Name of the block & Parameter name             & Value     \\ \hline
            ARDUINO\_SETUP    & Identifier of Arduino Card & 1         \\
                              & Serial com port number     & 2\portcmd \\ \hline
            TIME\_SAMPLE      & Duration of acquisition(s) & 10        \\
                              & Sampling period(s)         & 0.1       \\ \hline
            DCMOTOR\_SB       & Type of Shield             & 3         \\
                              & Arduino card number        & 1         \\ 
                              & PWM pin numbers            & 9 10      \\ 
                              & Motor number               & 1         \\ \hline
            STEP\_FUNCTION    & Step time                  & 5         \\
                              & Initial Value              & 100       \\
                              & Final Value                & 0         \\ \hline
          \end{tabular}
        \end{table}
        
        % Can you find out for how long the DC motor will run when this program
        % is executed?  In which block do we provide this information?  The
        % answer is that the DC motor will stop only when we terminate the
        % program.  As in the previous experiments, we can terminate the Xcos
        % program by pressing the stop button.  The DC motor gets released when
        % the stop button is pressed.
        
        
        
  \item Next, we will describe the Xcos code that drives the DC motor in
        both forward and reverse directions.  When the file required for
        this experiment is invoked, one gets the GUI as in
        \figref{fig:dcmotor-both}.  In the caption of this figure, one can
        see where to locate the file.
        
        \begin{figure}
          \centering
          \includegraphics[width=\smfig]{\LocDCMfig/dcmotor-both.png}
          \caption[Xcos control of the DC motor in forward and reverse
            directions]{Xcos control of the DC motor in forward and reverse
            directions.  This is what one sees when
            \LocDCMscibrief{dcmotor-both.zcos}, is invoked.}
          \label{fig:dcmotor-both}
        \end{figure}
        
        We will next explain how to set the parameters for this simulation.
        To set value on any block, one needs to right click and open the
          {\tt Block Parameters} or double click.  The values for each block
        is tabulated in \tabref{tab:dcmotor-both}.  All other parameters are
        to be left unchanged.
        \begin{table}
          \centering
          \caption{Xcos parameters to drive the DC motor in forward and
            reverse directions}
          \label{tab:dcmotor-both}
          \begin{tabular}{llc} \hline
            Name of the block & Parameter name             & Value     \\ \hline
            ARDUINO\_SETUP    & Identifier of Arduino Card & 1         \\
                              & Serial com port number     & 2\portcmd \\ \hline
            TIME\_SAMPLE      & Duration of acquisition(s) & 10        \\
                              & Sampling period(s)         & 0.1       \\ \hline
            DCMOTOR\_SB       & Type of Shield             & 3         \\
                              & Arduino card number        & 1         \\ 
                              & PWM pin numbers            & 9 10      \\ 
                              & Motor number               & 1         \\ \hline
            STEP\_FUNCTION    & Step time                  & 5         \\
                              & Initial Value              & 100       \\
                              & final value                & 0         \\ \hline
            CLOCK\_c          & Period                     & 1         \\
                              & Initialisation Time        & 0.1       \\ \hline
          \end{tabular}
        \end{table}
        
  \item Next, we will describe the Xcos code that drives the DC motor in
        a loop.  When the file required for
        this experiment is invoked, one gets the GUI as in
        \figref{fig:dcmotor-loop}.  In the caption of this figure, one can
        see where to locate the file.
        
        \begin{figure}
          \centering
          \includegraphics[width=\smfig]{\LocDCMfig/dcmotor-loop.png}
          \caption[Xcos control of the DC motor in forward and reverse
            directions]{Xcos control of the DC motor in forward and reverse
            directions.  This is what one sees when
            \LocDCMscibrief{dcmotor-loop.zcos}, is invoked.}
          \label{fig:dcmotor-loop}
        \end{figure}
        
        We will next explain how to set the parameters for this simulation.
        To set value on any block, one needs to right click and open the
          {\tt Block Parameters} or double click.  The values for each block
        is tabulated in \tabref{tab:dcmotor-loop}.  All other parameters are
        to be left unchanged.
        \begin{table}
          \centering
          \caption{Xcos parameters to drive the DC motor in a loop}
          \label{tab:dcmotor-loop}
          \begin{tabular}{llc} \hline
            Name of the block & Parameter name             & Value     \\ \hline
            ARDUINO\_SETUP    & Identifier of Arduino Card & 1         \\
                              & Serial com port number     & 2\portcmd \\ \hline
            TIME\_SAMPLE      & Duration of acquisition(s) & 10        \\
                              & Sampling period(s)         & 0.1       \\ \hline
            DCMOTOR\_SB       & Type of Shield             & 3         \\
                              & Arduino card number        & 1         \\
                              & PWM pin numbers            & 9 10      \\ 
                              & Motor number               & 1         \\ \hline
            STEP\_FUNCTION 1  & Step time                  & 3         \\
                              & Initial Value              & 100       \\
                              & Final Value                & 0         \\ \hline
            STEP\_FUNCTION 2  & Step time                  & 5         \\
                              & Initial Value              & 0         \\
                              & Final Value                & 100       \\ \hline
            STEP\_FUNCTION 3  & Step time                  & 7         \\
                              & Initial Value              & 0         \\
                              & Final Value                & 100       \\ \hline
            BIGSOM\_f         & Inputs ports signs/gain    & [1;-1;1]  \\ \hline
          \end{tabular}
        \end{table}
\end{enumerate}


%\section{Do we need any of these? \redcolor{Manas, please answer}}
%   \begin{figure}
%     \centering
%     \includegraphics[width=\smfig]{\LocDCMfig/dc-motor-simple.png}
%     \caption[Control of DC motor from Xcos]{Control of DC motor from
%       Xcos.  This is what one sees when {\tt
%         \LocDCMscibrief/dc-motor-simple.zcos} is invoked.}
%     \label{fig:dcm-xcos-simple}
%   \end{figure}



% \begin{enumerate}
% \item Card 1 on com 5 block: Right-click and open the block properties
%   or double click on this block.  In the resulting dialog window,
%   enter the com port number of your system.
% \item Typeshield 1 on card 1: Following the procedure mentioned above,
%   make sure that the entry for this block is 3, which corresponds to
%   L293D breakout board, as explained in \tabref{table:convention}. 
% \end{enumerate}
% Leave the other blocks unchanged.

% This Xcos program is used to put a specific PWM value to the DC motor.
% We can right click (or double click) on any block and see what
% parameter values are present in it.  By doing this, we can see that
% this program is used to put in a PWM value of 255.  Start executing
% the program by pressing the right arrow key.  

\begin{exercise} 
  Carry out the following exercise:
  \begin{enumerate}
    \item Keep reducing the PWM value and find out the minimum value
          required to run the DC motor.  Is this value in agreement with what
          we found in the previous section?
    \item Change the PWM value to $-100$ and check if the DC motor rotates
          in the opposite direction.
    \item Find out the smallest PWM value required to make the motor run
          in the opposite direction.  That is, find the least count for both
          directions.
    \item Come up with a method to rotate the motor in two directions for
          different time periods.
  \end{enumerate}
\end{exercise}

\section{Controlling the DC Motor from Python}
\subsection{Controlling the DC Motor}
In this section, we discuss how to carry out the experiments of the
previous section from Python.  We will list the same three experiments,
in the same order. As mentioned earlier, the shield must be removed from 
the \arduino\ and the \arduino\ needs to be connected to the computer 
with a USB cable, as shown in \figref{arduino}. The reader should go through the instructions given in
\secref{sec:python-start} before getting started.

\paragraph{Note:} The readers are advised to affix a small 
(very lightweight) piece of paper at the tip of the shaft of the DC motor. 
That will help them observe the direction of rotation 
of the DC motor while running the experiments. 

\begin{enumerate}
  \item In the first experiment, we will learn how to drive the DC motor
        from Python. The code for this experiment is 
        given in  \pyref{py:dcmotor-clock}. As explained earlier in \secref{sec:light-py}, we begin with 
        importing necessary modules followed by setting up the serial port. 
        Next, the code has a command of the following form: 
        \begin{lstlisting}[style=nonumbers]
              cmd_dcmotor_setup(1, H-Bridge type, Motor number, PWM pin 1, PWM pin 2)
        \end{lstlisting}
        As mentioned earlier, this chapter makes use of an H-Bridge circuit which 
        allows direction of the current passing through the DC motor to be changed.
        We are using L293D as an H-Bridge circuit in this book. Thus, we will pass the value 3 for
        H-Bridge type. The Python-Arduino toolbox, as explained in \secref{sec:python-toolbox}, 
        supports three types of H-Bridge circuit. \tabref{table:convention}
        provides the values to be passed for different H-Bridge circuits. 
        Next argument in the command given above is Motor number. Here, we pass the value 1. 
        Finally, we provide the PWM pins to which the DC motor is connected. As 
        shown in \figref{fig:dcmotorconn}, pins 9 and 10 are connected to the
        input of the breakout board. As a result, the command {\tt cmd\_dcmotor\_setup} becomes
        \lstinputlisting[firstline=27,lastline=27]
        {\LocDCMpycode/dcmotor-clock.py}
        
        The next line of \pyref{py:dcmotor-clock} is of the following form: 
        \begin{lstlisting}[style=nonumbers]
          cmd_dcmotor_run(1, Motor number, [sign] PWM value)
        \end{lstlisting}
        Here, we will pass the value 1 in Motor number.  As mentioned earlier, 
        for each of the PWM pins on \arduino\ board, the input can come from 8 bits.
        Thus, these pins can supply values between $- 255$ and $+ 255$. Positive values correspond to clockwise
        rotation while negative values correspond to anti-clockwise rotation. Based on the PWM value and polarity, 
        corresponding analog voltage is generated.  
        We put a PWM value of 100 to make the DC motor run at an intermediate speed.  
        As a result, the command {\tt cmd\_dcmotor\_run} becomes
        \lstinputlisting[firstline=28,lastline=28]
        {\LocDCMpycode/dcmotor-clock.py}
        
        The above-mentioned command does not say for how long the motor should run.  This is taken care of
        by the {\tt sleep} command, as given below:
        \lstinputlisting[firstline=29,lastline=29]{\LocDCMpycode/dcmotor-clock.py}
        With this, the DC motor will run for or 3 seconds. At last, 
        we release the DC motor, as shown below:
        \lstinputlisting[firstline=30,lastline=30]{\LocDCMpycode/dcmotor-clock.py}
        With the execution of this command, the PWM functionality on the \arduino\ pins
        is ceased.  This has the motor number as an input
        parameter. At last, we close the serial port. 
        
        
        \paragraph{Note:} If the sleep command (at line 29 of \pyref{py:dcmotor-clock}) 
        were not present, the DC motor will not even run: soon after putting the value 100, 
        the DC motor would be released, leaving no time in between.  On the other hand, if
        the DC motor is not released (\ie\ line number 30 of \pyref{py:dcmotor-clock} being commented), 
        the DC motor will go on rotating. That's why, it may be inferred that 
        line number 30 of \pyref{py:dcmotor-clock} is mandatory
        for every program. We encourage the readers to run  \pyref{sci:dcmotor-clock} by commenting
        any one or two of the lines numbered 29 and 30.  Go ahead and do it - you will not break
        anything.  At the most, you may have to unplug the USB cable connected to \arduino\ and
        restart the whole thing from the beginning.
        
  \item It is easy to make the DC motor run in the reverse direction by
        changing the sign of PWM value being written.  This is done in
        \pyref{py:dcmotor-both}.  In this code, we make the DC motor
        run in one direction for 3 seconds and then make it rotate in the
        reverse direction for 2 seconds.  The rotation in reverse direction
        is achieved by putting $- 100$ in the command {\tt cmd\_dcmotor\_run}, 
        as shown below:
        \lstinputlisting[firstline=28,lastline=28]
        {\LocDCMpycode/dcmotor-both.py}
        % This makes the green LED light up as well, recall the discussion in \secref{sec:led-pril}.  After
        After adding a {\tt sleep} of 2 seconds, we release the motor by issuing
        the command {\tt cmd\_dcmotor\_release}, followed by closing the serial port:
        \lstinputlisting[firstline=30,lastline=30]
        {\LocDCMpycode/dcmotor-both.py}
        With this, the motor comes to a halt.  
        % This turns the green LED off as well.
        
  \item Next, we make the DC motor run in forward and reverse
        directions, in a loop.  This is done through
        \pyref{py:dcmotor-loop}.  We first write PWM $+100$ for 3
        seconds.  After that, halt the motor for 2 seconds by writing zero PWM value.  
        Next, make the motor rotate in the reverse direction by writing PWM $-100$ for two seconds.  
        Next, we make the motor stop for one second. This procedure is put in a {\tt for} loop which runs for 4 iterations.
        At last, we release the motor by issuing the command {\tt cmd\_dcmotor\_release}, followed by closing the serial port
        
\end{enumerate}

% Initialization: In all the experiments in this section, we need to
% initialize the DC motor first, using a Python command of the following
% type:
% \begin{lstlisting}[style=nonumbers]
%   cmd_dcmotor_setup(1,H-Bridge type,Motor number,PWM pin 1,PWM pin 2)
% \end{lstlisting}

% As mentioned earlier, number 1 in the above list refers to the Arduino Uno board.
% We now discuss how to choose values for the other parameters in this command. As
% mentioned above, Popular numbering convention for different types of DC motor breakout
% boards is given in Table 7.1. For example, L293D is type 3. Next, we have to provide
% the motor number we want to control. In our case, it is number 1. Finally we want
% to provide PWM pin numbers on Arduino Uno. As mentioned earlier, we are using
% pins 10 and 11. In Table 7.2, we list the choices that we have made. Inserting these
% parameter values in the above shown Python command, we get the following command

% self.obj\_arduino.cmd\_dcmotor\_setup(1,3,1,self.pin1,self.pin2)

% To rotate the motor,we have to provide motor number
% and the PWM value. The Python command is of the form,

% cmd\_dcmotor\_run ( 1 , Motor number , ( sign ) (PWM value ) )

% The PWM values to be given are as same as explained in Scilab code before.

% To run the motor for specified amount of time,we will use sleep command

% sleep(3) //sleep for 3 seconds

% To release the dc motor, we will use the following command

% cmd\_dcmotor\_release(1,1) //Motor 1 is release

% To run motor in loop, for loop is used in Python code 7.3.

\subsection{Python Code}
\label{sec:dcmotor-python-code}
\addtocontents{pyd}{\protect\addvspace{\codclr}}

\begin{pycode}
  \pcaption{Rotating the DC motor}
  {Rotating the DC motor.  Available at
    \LocDCMpybrief{dcmotor-clock.py}.}
  \label{py:dcmotor-clock}
  \lstinputlisting{\LocDCMpycode/dcmotor-clock.py}
\end{pycode}

\begin{pycode}
  \pcaption{Rotating the DC motor in both directions}
  {Rotating DC motor in both directions.  Available at
    \LocDCMpybrief{dcmotor-both.py}.}
  \label{py:dcmotor-both}
  \lstinputlisting{\LocDCMpycode/dcmotor-both.py}
\end{pycode}

\begin{pycode}
  \pcaption{Rotating the DC motor in both directions in a loop}{Rotating
    the DC motor in both directions in a loop.
    Available at
    \LocDCMpybrief{dcmotor-loop.py}.}
  \label{py:dcmotor-loop}
  \lstinputlisting{\LocDCMpycode/dcmotor-loop.py}
\end{pycode}

\section{Controlling the DC Motor from Julia}
\subsection{Controlling the DC Motor}
In this section, we discuss how to carry out the experiments of the
previous section from Julia.  We will list the same three experiments,
in the same order. As mentioned earlier, the shield must be removed from 
the \arduino\ and the \arduino\ needs to be connected to the computer 
with a USB cable, as shown in \figref{arduino}. The reader should go through the instructions given in \secref{sec:julia-start} before getting started.

\paragraph{Note:} The readers are advised to affix a small 
(very lightweight) piece of paper at the tip of the shaft of the DC motor. 
That will help them observe the direction of rotation 
of the DC motor while running the experiments.

\begin{enumerate}
  \item In the first experiment, we will learn how to drive the DC motor
        from Julia. The code for this experiment is 
        given in  \juliaref{julia:dcmotor-clock}. 
        As explained earlier in \secref{sec:light-julia}, we begin with 
        importing necessary modules followed by setting up the serial port. 
        Next, the code has a command of the following form: 
        \begin{lstlisting}[style=nonumbers]
          DCMotorSetup(1, H-Bridge type, Motor number, PWM pin 1, PWM pin 2)
        \end{lstlisting}
        As mentioned earlier, this chapter makes use of an H-Bridge circuit which 
        allows direction of the current passing through the DC motor to be changed.
        We are using L293D as an H-Bridge circuit in this book. Thus, we will pass the value 3 for
        H-Bridge type. The Julia-Arduino toolbox, as explained in \secref{sec:julia-toolbox}, 
        supports three types of H-Bridge circuit. \tabref{table:convention}
        provides the values to be passed for different H-Bridge circuits. 
        Next argument in the command given above is Motor number. Here, we pass the value 1. 
        Finally, we provide the PWM pins to which the DC motor is connected. As 
        shown in \figref{fig:dcmotorconn}, pins 9 and 10 are connected to the
        input of the breakout board. As a result, the command {\tt DCMotorSetup} becomes
        \lstinputlisting[firstline=5,lastline=5]
        {\LocDCMjuliacode/dcmotor-clock.jl}
        
        The next line of \juliaref{julia:dcmotor-clock} is of the following form: 
        \begin{lstlisting}[style=nonumbers]
          DCMotorRun(1, Motor number, [sign] PWM value)
        \end{lstlisting}
        Here, we will pass the value 1 in Motor number.  As mentioned earlier, 
        for each of the PWM pins on \arduino\ board, the input can come from 8 bits.
        Thus, these pins can supply values between $- 255$ and $+ 255$. Positive values correspond to clockwise
        rotation while negative values correspond to anti-clockwise rotation. Based on the PWM value and polarity, 
        corresponding analog voltage is generated.  
        We put a PWM value of 100 to make the DC motor run at an intermediate speed.  
        As a result, the command {\tt DCMotorRun} becomes
        \lstinputlisting[firstline=6,lastline=6]
        {\LocDCMjuliacode/dcmotor-clock.jl}
        
        The above-mentioned command does not say for how long the motor should run.  This is taken care of
        by the {\tt sleep} command, as given below:
        \lstinputlisting[firstline=7,lastline=7]{\LocDCMjuliacode/dcmotor-clock.jl}
        With this, the DC motor will run for or 3 seconds. At last, 
        we release the DC motor, as shown below:
        \lstinputlisting[firstline=8,lastline=8]{\LocDCMjuliacode/dcmotor-clock.jl}
        With the execution of this command, the PWM functionality on the \arduino\ pins
        is ceased.  This has the motor number as an input
        parameter. At last, we close the serial port. 
        
        \paragraph{Note:} If the sleep command (at line 7 of \juliaref{julia:dcmotor-clock}) 
        were not present, the DC motor will not even run: soon after putting the value 100, 
        the DC motor would be released, leaving no time in between.  On the other hand, if
        the DC motor is not released (\ie\ line number 8 of \juliaref{julia:dcmotor-clock} being commented), 
        the DC motor will go on rotating. That's why, it may be inferred that 
        line number 8 of \juliaref{julia:dcmotor-clock} is mandatory
        for every program. We encourage the readers to run  \juliaref{julia:dcmotor-clock} by commenting
        any one or two of the lines numbered 7 and 8.  Go ahead and do it - you will not break
        anything.  At the most, you may have to unplug the USB cable connected to \arduino\ and
        restart the whole thing from the beginning.
        
  \item It is easy to make the DC motor run in the reverse direction by
        changing the sign of PWM value being written.  This is done in
        \juliaref{julia:dcmotor-both}.  In this code, we make the DC motor
        run in one direction for 3 seconds and then make it rotate in the
        reverse direction for 2 seconds.  The rotation in reverse direction
        is achieved by putting $- 100$ in the command {\tt DCMotorRun}, 
        as shown below:
        \lstinputlisting[firstline=8,lastline=8]
        {\LocDCMjuliacode/dcmotor-both.jl}
        % This makes the green LED light up as well, recall the discussion in \secref{sec:led-pril}.  After
        After adding a {\tt sleep} of 2 seconds, we release the motor by issuing
        the command {\tt DCMotorRelease}, followed by closing the serial port:
        \lstinputlisting[firstline=10,lastline=11]
        {\LocDCMjuliacode/dcmotor-both.jl}
        With this, the motor comes to a halt.  
        % This turns the green LED off as well.
        
  \item Next, we make the DC motor run in forward and reverse
        directions, in a loop.  This is done through
        \juliaref{julia:dcmotor-loop}.  We first write PWM $+100$ for 3
        seconds.  After that, halt the motor for 2 seconds by writing zero PWM value.  
        Next, make the motor rotate in the reverse direction by writing PWM $-100$ for two seconds.  
        Next, we make the motor stop for one second. This procedure is put in a {\tt for} loop which runs for 4 iterations.
        At last, we release the motor by issuing the command {\tt DCMotorRelease}, followed by closing the serial port. 
        
\end{enumerate}


\subsection{Julia Code}
\label{sec:dcmotor-julia-code}
\addtocontents{juliad}{\protect\addvspace{\codclr}}

\begin{juliacode}
  \jcaption{Rotating the DC motor}
  {Rotating the DC motor.  Available at
    \LocDCMjuliabrief{dcmotor-clock.jl}.}
  \label{julia:dcmotor-clock}
  \lstinputlisting{\LocDCMjuliacode/dcmotor-clock.jl}
\end{juliacode}

\begin{juliacode}
  \jcaption{Rotating the DC motor in both directions}
  {Rotating DC motor in both directions.  Available at
    \LocDCMjuliabrief{dcmotor-both.jl}.}
  \label{julia:dcmotor-both}
  \lstinputlisting{\LocDCMjuliacode/dcmotor-both.jl}
\end{juliacode}

\begin{juliacode}
  \jcaption{Rotating the DC motor in both directions in a loop}{Rotating
    the DC motor in both directions in a loop.
    Available at
    \LocDCMjuliabrief{dcmotor-loop.jl}.}
  \label{julia:dcmotor-loop}
  \lstinputlisting{\LocDCMjuliacode/dcmotor-loop.jl}
\end{juliacode}


\section{Controlling the DC Motor from OpenModelica}
\subsection{Controlling the DC Motor}
In this section, we discuss how to carry out the experiments of the
previous section from OpenModelica.  We will list the same three experiments,
in the same order.  As mentioned earlier, the shield must be removed from 
the \arduino\ and the \arduino\ needs to be connected to the computer 
with a USB cable, as shown in \figref{arduino}. The reader should go through the instructions given in
\secref{sec:OpenModelica-start} before getting started.



\paragraph{Note:} The readers are advised to affix a small 
(very lightweight) piece of paper at the tip of the shaft of the DC motor. 
That will help them observe the direction of rotation 
of the DC motor while running the experiments.

\begin{enumerate}
  \item In the first experiment, we will learn how to drive the DC motor
        from OpenModelica. The code for this experiment is 
        given in  \OpenModelicaref{OpenModelica:dcmotor-clock}. 
        As explained earlier in \secref{sec:light-OpenModelica}, 
        we begin with importing the two packages: Streams and SerialCommunication followed 
        by setting up the serial port. Next, the code has a command of the following form: 
        \begin{lstlisting}[style=nonumbers]
          cmd_dcmotor_setup(1, H-Bridge type, Motor number, PWM pin 1, PWM pin 2)
        \end{lstlisting}
        As mentioned earlier, this chapter makes use of an H-Bridge circuit which 
        allows direction of the current passing through the DC motor to be changed.
        We are using L293D as an H-Bridge circuit in this book. Thus, we will pass the value 3 for
        H-Bridge type. The OpenModelica-Arduino toolbox, as explained in 
        \secref{sec:load-om-toolbox}, 
        supports three types of H-Bridge circuit. \tabref{table:convention}
        provides the values to be passed for different H-Bridge circuits. 
        Next argument in the command given above is Motor number. Here, we pass the value 1. 
        Finally, we provide the PWM pins to which the DC motor is connected. As 
        shown in \figref{fig:dcmotorconn}, pins 9 and 10 are connected to the
        input of the breakout board. As a result, the command {\tt cmd\_dcmotor\_setup} becomes
        \lstinputlisting[firstline=15,lastline=15]
        {\LocDCMOpenModelicacode/dcmotor-clock.mo}
        
        The next line of \OpenModelicaref{OpenModelica:dcmotor-clock} is of the following form: 
        \begin{lstlisting}[style=nonumbers]
          cmd_dcmotor_run(1, Motor number, [sign] PWM value)
        \end{lstlisting}
        Here, we will pass the value 1 in Motor number.  As mentioned earlier, 
        for each of the PWM pins on \arduino\ board, the input can come from 8 bits.
        Thus, these pins can supply values between $- 255$ and $+ 255$. Positive values correspond to clockwise
        rotation while negative values correspond to anti-clockwise rotation. Based on the PWM value and polarity, 
        corresponding analog voltage is generated.  
        We put a PWM value of 100 to make the DC motor run at an intermediate speed.  
        As a result, the command {\tt cmd\_dcmotor\_run} becomes
        \lstinputlisting[firstline=16,lastline=16]
        {\LocDCMOpenModelicacode/dcmotor-clock.mo}
        
        The above-mentioned command does not say for how long the motor should run.  This is taken care of
        by the {\tt sleep} command, as given below:
        \lstinputlisting[firstline=17,lastline=17]{\LocDCMOpenModelicacode/dcmotor-clock.mo}
        With this, the DC motor will run for 3000 milliseconds or 3 seconds. At last, 
        we release the DC motor, as shown below:
        \lstinputlisting[firstline=18,lastline=18]{\LocDCMOpenModelicacode/dcmotor-clock.mo}
        With the execution of this command, the PWM functionality on the \arduino\ pins
        is ceased.  This has the motor number as an input
        parameter. At last, we close the serial port. 
        
        \paragraph{Note:} If the sleep command (at line 17 of \OpenModelicaref{OpenModelica:dcmotor-clock}) 
        were not present, the DC motor will not even run: soon after putting the value 100, 
        the DC motor would be released, leaving no time in between.  On the other hand, if
        the DC motor is not released (\ie\ line number 18 of \OpenModelicaref{OpenModelica:dcmotor-clock} being commented), 
        the DC motor will go on rotating. That's why, it may be inferred that 
        line number 18 of \OpenModelicaref{OpenModelica:dcmotor-clock} is mandatory
        for every program. We encourage the readers to run  \OpenModelicaref{OpenModelica:dcmotor-clock} by commenting
        any one or two of the lines numbered 17 and 18.  Go ahead and do it - you will not break
        anything.  At the most, you may have to unplug the USB cable connected to \arduino\ and
        restart the whole thing from the beginning.
        
  \item It is easy to make the DC motor run in the reverse direction by
        changing the sign of PWM value being written.  This is done in
        \OpenModelicaref{OpenModelica:dcmotor-both}.  In this code, we make the DC motor
        run in one direction for 3 seconds and then make it rotate in the
        reverse direction for 2 seconds.  The rotation in reverse direction
        is achieved by putting $- 100$ in the command {\tt cmd\_dcmotor\_run}, 
        as shown below:
        \lstinputlisting[firstline=17,lastline=17]
        {\LocDCMOpenModelicacode/dcmotor-both.mo}
        % This makes the green LED light up as well, recall the discussion in \secref{sec:led-pril}.  After
        After adding a {\tt sleep} of 2 seconds, we release the motor by issuing
        the command {\tt cmd\_dcmotor\_release}, followed by closing the serial port:
        \lstinputlisting[firstline=19,lastline=19]
        {\LocDCMOpenModelicacode/dcmotor-both.mo}
        With this, the motor comes to a halt.  
        % This turns the green LED off as well.
        
  \item Next, we make the DC motor run in forward and reverse
        directions, in a loop.  This is done through
        \OpenModelicaref{OpenModelica:dcmotor-loop}.  We first write PWM $+100$ for 3
        seconds.  After that, halt the motor for 2 seconds by writing zero PWM value.  
        Next, make the motor rotate in the reverse direction by writing PWM $-100$ for two seconds.  
        Next, we make the motor stop for one second. This procedure is put in a {\tt for} loop which runs for 4 iterations.
        At last, we release the motor by issuing the command {\tt cmd\_dcmotor\_release}, followed by closing the serial port. 
        
\end{enumerate}



%%%%%%%OpenModelica description ends

% \subsection{Using the Xcos features in DC motor control}
% Xcos can be used to control the DC motor in many different ways.  In
% this section, we will see a few approaches.  First, we will make the
% DC motor start and stop.  For this, we will open the program given in \figref{fig:dcm-xcos-start-stop}.
% \begin{figure}
% \centering
% \includegraphics[width=\smfig]{\LocDCMfig/dc-motor-start-stop.png}
% \caption[Xcos program to make the DC motor to rotate, pause and
%   repeat]{Xcos program to make the DC motor to rotate, pause and
%   repeat.  This is what one sees when {\tt
%     \LocDCMscibrief/dc-motor-start-stop.zcos} is invoked.}
% \label{fig:dcm-xcos-start-stop}
% \end{figure}

% The input is a train of pulses of positive values.  The amplitude of
% these pulses is chosen to be 255.  The period is chosen to be 1s.
% \redcolor{these values have to be checked - this section to be
%   rewritten.}  On executing this Xcos code, the DC motor rotates for
% 1s, pauses for 1s, and repeats.  This continues until the Xcos program
% is terminated.

% \begin{exercise}
% Carry out the following exercise:
% \begin{enumerate}
% \item The user may repeat this exercise with different amplitude and
%   periods. 
% \item The above may be repeated with negative PWM values.
% \item One may find the least count for all experiments.
% \end{enumerate}
% \end{exercise}

% We will next see how to give positive and negative values of PWM in
% the same experiment.  For this, we will use the Xcos program given in
% \figref{fig:dcm-xcos-both}.  This code is similar to the previous one,
% but for a few minor changes.  First of all, we have introduced a gain
% block, and assigned a value of 255.  This block is excited by pulses
% of $+1$ and $-1$ alternately.  This simple approach, however, results
% in the DC motor running in both directions for identical durations.
% \begin{figure}
% \centering
% \includegraphics[width=\lgfig]{\LocDCMfig/dc-motor-both.png}
% \caption[Xcos program to make the DC motor to rotate, pause and to
%   rotate in the opposite direction]{Xcos program to make the DC motor
%   to rotate, pause and to rotate in the opposite direction.  This is
%   what one sees when {\tt \LocDCMscibrief/dc-motor-both.zcos} is
%   invoked.}
% \label{fig:dcm-xcos-both}
% \end{figure}

% \begin{exercise}
% Carry out the following exercise:
% \begin{enumerate}
% \item Repeat this experiment for different amplitudes and periods.
% \end{enumerate}
% \end{exercise}


% \subsection{Troubleshooting \redcolor{Do we need this? - Manas, please answer}}
% \begin{enumerate}
% \item If we want to connect external supply, Ground (Gnd) pin of L293D board, external supply and \arduino\ board should be shorted. This creates common ground voltage for entire set-up.
% \item We can connect more than one motor simultaneously using H-bridge like L293D. Typically break-out boards support 2 motors. However we are limited by number of PWM pins available on the board. For each motor cable of bidirectional motion, we need two PWM pins
% \end{enumerate}


\subsection{OpenModelica Code}
\label{sec:dcmotor-OpenModelica-code}
Unlike other code files, the code/ model for running experiments using OpenModelica are 
available inside the OpenModelica-Arduino toolbox, as explained in \secref{sec:load-om-toolbox}.
Please refer to \figref{om-examples-toolbox} to know how to locate the experiments. 

\addtocontents{OpenModelicad}{\protect\addvspace{\codclr}}

\begin{OpenModelicacode}
  \mcaption{Rotating the DC motor}
  {Rotating the DC motor.  
  Available at Arduino -> SerialCommunication -> Examples -> dcmotor 
    -> dcmotor\_clock.}
  \label{OpenModelica:dcmotor-clock}
  \lstinputlisting{\LocDCMOpenModelicacode/dcmotor-clock.mo}
\end{OpenModelicacode}

\begin{OpenModelicacode}
  \mcaption{Rotating the DC motor in both directions}
  {Rotating DC motor in both directions.  Available at Arduino -> SerialCommunication -> Examples -> dcmotor 
    -> dcmotor\_both.}
  \label{OpenModelica:dcmotor-both}
  \lstinputlisting{\LocDCMOpenModelicacode/dcmotor-both.mo}
\end{OpenModelicacode}

\begin{OpenModelicacode}
  \mcaption{Rotating the DC motor in both directions in a loop}{Rotating
    the DC motor in both directions in a loop.
    Available at Arduino -> SerialCommunication -> Examples -> dcmotor 
    -> dcmotor\_loop.}
  \label{OpenModelica:dcmotor-loop}
  \lstinputlisting{\LocDCMOpenModelicacode/dcmotor-loop.mo}
\end{OpenModelicacode}
%%%%%%%%%%OpenModelica code ends
