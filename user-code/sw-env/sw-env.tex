\chapter{Communcation between Software and Arduino}
\thispagestyle{empty}
\label{sec:sw-env}

\newcommand{\LocSWfig}{\Origin/user-code/sw-env/figures}
\newcommand{\LocSWscicode}{\Origin/user-code/sw-env/scilab}
\newcommand{\LocSWscibrief}[1]{{\texttt
                  Origin/user-code/sw-env/scilab/#1}, see \fnrefp{fn:file-loc}}
\newcommand{\LocSWardcode}{\Origin/user-code/sw-env/arduino}
\newcommand{\LocSWardbrief}[1]{{\tt \seqsplit{
                        Origin/user-code/sw-env/arduino/#1}}, see \fnrefp{fn:file-loc}}

\newcommand{\LocSWchkcode}{\Origin/tools}
\newcommand{\LocSWchkbrief}[1]{{\tt \seqsplit{
                        Origin/tools/#1}}, see \fnrefp{fn:file-loc}}
\newcommand{\LocSWfirmcode}{\Origin/tools/arduino-firmware}
\newcommand{\LocSWfirmbrief}[1]{{\tt \seqsplit{
                        Origin/tools/arduino-firmware/#1}}, see \fnrefp{fn:file-loc}}

\newcommand{\LocFIMpycode}{\Origin/user-code/led/python}  %added for python
\newcommand{\LocFIMpybrief}[1]{{\tt \seqsplit{%
                        Origin/user-code/led/python/#1}}, see \fnrefp{fn:file-loc}} % added for python


\newcommand{\LocFIMjuliacode}{\Origin/user-code/led/julia}  %added for julia
\newcommand{\LocFIMjuliabrief}[1]{{\tt \seqsplit{%
                        Origin/user-code/led/julia/#1}}, see \fnrefp{fn:file-loc}} % added for julia

%%%%%%OpenModelica Starts
\newcommand{\LocFIMOpenModelicacode}{\Origin/user-code/led/OpenModelica}  %added for OpenModelica
\newcommand{\LocFIMOpenModelicabrief}[1]{{\tt \seqsplit{%
                        Origin/user-code/led/OpenModelica/#1}}, see \fnrefp{fn:file-loc}} % added for OpenModelica

%%%%%OpenModelica Ends


In this chapter, we shall briefly walk through the software
environment that needs to be set up before we could start with the
\arduino\ board based experiments. We shall start with the \arduino\
compatible Integrated Development Environment (IDE), termed as Arduino
IDE, that would be used to load the firmware on to the
microcontroller. The firmware to be loaded could be developed to serve
different purposes as per the requirement. For example, 
\begin{itemize}
      \item To run \arduino\ stand alone, without waiting for any commands
            from other software or hardware, for the specified time or until
            power off
      \item To decode the commands sent by other software, such as \scilab,
            through serial port, and execute the given instructions
            %\item Combination of the above two
\end{itemize}
Next, we shall discuss Scilab and Xcos, which are open source software
tools, and a related toolbox that can communicate with \arduino\ 
over a serial port using RS232 protocol.

\section{Arduino IDE}\label{arduino-ide}
\label{sec:ard-start}
Arduino development environment is compatible with popular desktop
operating systems. In this section, we will learn to set up this tool
for the computers running Microsoft Windows or Linux. Later, we shall
explore the important menu options in the Arduino IDE and run a sample
program.  Following two steps have to be followed whatever operating
system is used:

\begin{enumerate}
      \item To begin, we need an \arduino\ board with a USB cable (A plug to
            B plug) as shown in the \figref{arduino}.
      \item Connect it to a computer and power it up. The moment you connect \arduino\
            to the computer, an on-board LED will turn ON.
\end{enumerate}

\subsection{Downloading and installing on Windows}
First carry out the steps numbered 1 and 2 given above.
Starting from download, we shall go through the steps to set up
Arduino IDE on Windows OS:

\begin{enumerate}
      \setcounter{enumi}2
      \item Visit the URL, \\{\tt https://www.arduino.cc/en/software} \\
            locate on the right side of the page the link \emph{Windows ZIP file} and click it.  
            This may redirect you to download/donate page. Read the instructions and proceed with the
            download.
      \item Extract the downloaded ZIP file to Desktop. Do not alter any
            file or directory structure.
      \item Click on the Windows Start Menu, and open up the ``Control
            Panel''.
      \item While in the Control Panel, navigate to ``System and Security'',
            click on ``System'' and then choose the ``Device Manager''.
      \item Look for ``Other devices'' in the ``Device Manager'' list,
            expand and locate ``Unknown device''.  This may be similar to what
            is shown in \figref{win-device-manager}.
      \item Right click on the ``Unknown device'' and select the ``Update
            Driver Software'' option as shown in \figref{win-dri-update}.
      \item Next, choose the ``Browse my computer for Driver software''
            option.
      \item Navigate to newly extracted Arduino folder on the Desktop and
            select ``drivers'' folder.
      \item Windows will now finish the driver installation. The Arduino IDE
            is ready for use.
\end{enumerate}

\begin{figure}
      \centering
      \includegraphics[width=\linewidth]{\LocHWfig/hw-device-manager.jpg}
      \caption{Windows device manager}
      \label{win-device-manager}
\end{figure}

\begin{figure}
      \centering
      \includegraphics[width=\linewidth]{\LocHWfig/update-driver.png}
      \caption{Windows update driver option}
      \label{win-dri-update}
\end{figure}


\subsection{Downloading and installing on GNU/Linux Ubuntu}
We will now explain the installation of Arduino software on the
GNU/Linux operating system. We shall perform the installation on the 64-bit 
Ubuntu 18.04 LTS operating system.  These
instructions will work for other GNU distributions too, with little or
no modification.  First carry out the steps numbered 1 and 2 given
above.  Then carry out the following:

\begin{enumerate}
      \setcounter{enumi}2
      \item First, update your system. Open the terminal emulator, type, 
      {\tt sudo apt-get update} and press Enter. 
      \item Find out your operating system support for 64 bit
            instructions. Open the terminal emulator and type, {\tt uname -m}
      \item If it returns ``x86\_64'', then your computer has 64-bit
            operating system.   There is no visible performance difference in 32
            and 64-bit Arduino versions.
      \item Download the suitable Arduino Software version (32 or 64 bits)
            from \\ {\tt https://www.arduino.cc/en/software}.  As mentioned
            earlier, we will perform experiments with a 64-bit installation.
            
      \item At the time of writing this book, we worked with version 1.8.13.
            Assuming that you have downloaded tar file in~
            {\large \textasciitilde}/Downloads directory, perform the following
            steps on the terminal:
            \begin{quote}
                  {\tt cd {\large\textasciitilde}/Downloads\\
                  tar -xvf arduino-1.8.13-linux64.tar.xz\\
                        sudo mv arduino-1.8.13 /opt}
            \end{quote}
            
      \item In the same terminal session, install the required Java Runtime
            Environment with a command like,
            {\tt sudo apt-get -y install openjdk-8-jre}
            
      \item \label{itm:port-check} Execute the
            following command on the terminal to list the serial port number.\\
            {\tt ls /dev/ttyACM*}\\
            Note down the serial device filename.  Suppose that it
            is {\tt ttyACM0}.
      \item \label{itm:port-access} To make USB port available to all users, set the read write
            permission to the listed port:
            {\tt sudo chmod a+rw /dev/ttyACM0}. Each time you plug the \arduino\
            into the computer, you need to execute the commands given in the steps 
            numbered \ref{itm:port-check} and \ref{itm:port-access}. 
            
      \item \label{itm:create-shortcut} Create a shortcut on the desktop:\\
            {\tt cd {\large \textasciitilde}/Desktop\\
            ln -s /opt/arduino-1.8.13/arduino}
      \item \label{itm:give-permission} Give executable permission to this file through the following
            command on the terminal: {\tt chmod +x arduino}
            %   Ubuntu opens executable text files with an editor instead of
            %   executing them. To be able execute a file, open the ``Files''
            %   program from the launcher, go to menu ``Edit'', ``Preferences'', tab
            %   ``Behavior'' and set ``Executable Text Files'' to ``Ask each time'',
            %   as shown in \figref{ard-lin-executable}.
            
\end{enumerate}
% \begin{figure}
%   \centering
%   \includegraphics[scale=0.5]{\LocHWfig/executable.png}
%   \caption{Executable permission to Arduino IDE}
%   \label{ard-lin-executable}
% \end{figure}
Then double click the Arduino shortcut on the Desktop and, click ``Run''
in the dialog window to start the Arduino IDE. The dialog box is shown in \figref{ard-lin-run} for reference.
\begin{figure}
      \centering
      \includegraphics[scale=0.5]{\LocHWfig/run.png}
      \caption{Confirmation for executing Arduino script}
      \label{ard-lin-run}
\end{figure}
The Arduino IDE is now ready for use. There are chances that you might not 
get the Arduino shortcut on your Desktop after executing the commands given in 
steps numbered \ref{itm:create-shortcut} and \ref{itm:give-permission}. 
In that case, you can navigate into the {\tt /opt/} directory and execute the 
commands as given in Fig. to start the Arduino IDE. 


\subsection{Arduino Development Environment}
\label{sec:Arduino-IDE}
The Arduino development environment, as shown in \figref{ard-ide},
consists of 
a text editor for writing code, a message area, a text console, a
toolbar with buttons for common functions, and a series of menus. It
connects to the Arduino hardware to upload programs and communicate
with them.

\begin{figure}
      \centering
      \includegraphics[width=\linewidth]{\LocHWfig/arduino-ide.jpg}
      \caption{Arduino IDE}
      \label{ard-ide}
\end{figure}
Software written using Arduino are called sketches. These sketches are
written in the text editor. Sketches are saved with the file extension
``.ino''. The frequently used icons shown in the toolbar, below the menu bar, are explained next. The names of these icons can be viewed by hovering the mouse pointer over each of them.

\begin{enumerate}
      \item Verify: Checks your code for errors
      \item Upload: Compiles your code and uploads it to the Arduino I/O
            board
      \item New: Creates a new sketch
      \item Open: Presents a menu of all the sketches in your
            sketchbook - clicking one will open it within the current window
      \item Save: Saves your sketch
      \item Serial Monitor: Opens the serial port window - the location of
            this is shown in the top right hand corner of \figref{ard-ide}
\end{enumerate}
Note that these appear from left to right in the editor window. Next, we shall go through the additional useful options under menu.
\begin{enumerate}
      \item File
            \begin{enumerate}
                  \item Examples: Examples that come at the time of installation
                  \item Page Setup: Configures the page parameters for printer
                  \item Preferences: Customizes font, language and other parameters for
                        the IDE
            \end{enumerate}
      \item Sketch
            \begin{enumerate}
                  \item Include Library: Adds a library to your sketch by inserting {\tt
                                    \#include} statements at the start of your code
            \end{enumerate}
      \item Tools
            \begin{enumerate}
                  \item Auto Format: Indents code so that opening and closing curly
                        braces line up
                  \item Archive Sketch: Archives a copy of the current sketch in .zip
                        format. The archive is placed in the same directory as the sketch.
                  \item Board: Selects the board that you're using
                  \item Port: This menu contains all the serial devices (real or
                        virtual) on your machine. It should automatically refresh every time
                        you open the top-level tools menu.
                  \item Programmer: This can be used to select a hardware programmer when programming a board or chip and not using the onboard USB-serial
                        connection. Normally you won't need this, but if you're burning a
                        bootloader to a new microcontroller, you will use this.
                  \item Burn Bootloader: The items in this menu allow you to burn a
                        bootloader onto the microcontroller on an Arduino board. This is not
                        required for normal use of an Arduino board but is useful if you
                        purchase a new ATmega microcontroller (which normally comes without a
                        bootloader). Ensure that you've selected the correct board from the
                        Boards menu before burning the bootloader.
            \end{enumerate}
\end{enumerate}

\subsection{Testing Arduino with a sample program}
\label{sec:testing-arduino}
Now, as we have a basic understanding of Arduino IDE, let us try an
example program.
\begin{enumerate}
      \item Open the Arduino IDE by clicking the shortcut ``arduino'' from
            Desktop in Ubuntu. In MS Windows browse to extracted Arduino folder
            on Desktop and double click on ``arduino.exe''.
      \item In the Arduino IDE, to know the path of your sketch files,
            navigate to File, then Preferences and then locate the ``Sketchbook
            location'' text box at the top.  You may change the path of your
            storage location. In this book we will keep it unchanged. The path
            will be different for Windows and Ubuntu.
      \item To load a sample program, navigate and click on sketch ``File'',
            then Examples, then 01.Basics, and then Blink.
      \item A new IDE instance will open with Blink LED code.  You may close
            the previous IDE window now.
      \item Click ``verify'' to compile. The ``status bar'' below text editor
            shall show ``Done compiling'' on success.
      \item Connect Arduino UNO board to PC. You may connect the board
            before writing the sketch too.
      \item Now, navigate to ``Tools'', then Port and select the available
            port. If port option is greyed out (or disabled) then reinsert the
            USB cable to PC.
      \item Now select the upload button to compile and send the firmware to
            the Arduino Uno board.
      \item If the upload is successful, you will notice the onboard orange LED
            next to the Arduino logo will start blinking.
      \item It is safe to detach USB cable at any moment.
\end{enumerate}

Arduino programming syntax is different from other languages. In an
embedded set up, a program is expected to run forever. To facilitate
this, Arduino programming structure has two main functions: setup():
Used to initialize variables, pin modes, libraries etc. The setup
function will run only once after each power up or board reset.
loop(): Code inside this function runs forever. An Arduino program
must have setup() and loop() functions.  We will give several examples
in this book to explain this usage.

An inbuilt offline help is available within the IDE. You may access
explanation on IDE by navigating to ``Help'' and then
Environment. Access Arduino language reference at ``Help'' and then
Reference. Access FAQs and troubleshooting tricks at ``Help'' and then
Troubleshooting.

\subsection{Firmware}
We have provided a code to check whether the Arduino firmware has been
properly installed.  The first few lines of this code follow. 

\begin{ardcode}
      \acaption{First 10 lines of the Arduino firmware}{First 10 lines of
            the Arduino firmware.  Available at
            \LocSWfirmbrief{arduino-firmware.ino}. By following the
            steps given in sections \ref{sec:Arduino-IDE} and 
            \ref{sec:testing-arduino}, open this code in 
            Arduino IDE and upload it to Arduino Uno. 
            Once the upload is successful, you should expect a message 
            in the bottom of Arduino IDE as shown in \figref{ard-ide}. }
      \label{ard:firmware}
      \lstinputlisting[firstline=1,lastline=10]
      {\LocSWfirmcode/arduino-firmware.ino}
\end{ardcode}

% \subsection{Arduino firmware to work with scilab toolbox}
% \label{sec:firmware}
% A firmware is basically a program that continuously runs inside a
% microcontroller. It is a collection of routines corresponding to the
% required functionalities. It is typically written in Assembly and C
% programming language. It is compiled and converted into
% binary(hexadecimal values with addresses) for the target
% microcontroller. The binary file(also called hex file) is then
% uploaded to the microcontroller’s internal ROM. The firmware that has to
% be used to work with scilab toolbox is at \ardref{ard:firmware}.  It
% is an Arduino IDE compatible file and can be opened in an Arduino
% IDE. Let us see a brief explanation of this firmware.

% The firmware used for Arduino Uno in this book has the following tasks
% to perform:
% \begin{enumerate}
% \item Reading instructions from a computer(running Scilab) over serial
%   interface and decoding them. 
% \item Performing the task mentioned in the instruction.
% \item Optionally sending data back to the computer over an serial
%   interface.
% \end{enumerate}
% Let us see a simple example of reading values from the LDR that is
% on the shield.

% The firmware waits for a particular character (command) to be sent
% from the computer. The character “A”, in quoted form, as shown here,
% is reserved for analog read
% routine. So if Scilab wants analog values from the microcontroller, it
% sends the character “A” to Arduino Uno. On receiving “A”, the Arduino
% Uno 
% jumps to the routine of Analog read. Here it again waits for the
% computer to now send the pin number from where it is supposed to read
% the LDR value. This pin number is in ASCII text. Arduino Uno first
% checks if the ASCII lies between a valid range. If yes, it takes its
% ASCII value as a valid pin number. The value 48 is subtracted from it
% to reveal the character and thus the pin number. This pin number is
% then sent to the analogRead() function. The analogRead() function is
% an inbuilt Arduino function imported from the header file. The
% analogRead() function then actually reads from the pin and returns the
% analog value. This value is then sent back to the
% computer(Scilab). The correct firmware must be loaded inside Arduino
% Uno to be able to successfully carry out any of the experiment
% explained throughout this book. It is strongly recommended to confirm
% this before proceeding. 

\section{Scilab}
\label{sec:sci-start}
Scilab is a free and open source computing software for science and
engineering applications \cite{scilab-ref}. It is released under GPL
compatible CeCILL licence.  It uses the state of the art linear
algebra package LAPACK, just as in Matlab.  Scilab has hundreds of
inbuilt functions which cater to a variety of areas such as signal
processing, control system design, statistics, optimization and many
more. It has 2D and 3D visualisation capabilities for generating
excellent plots. It provides Matlab binary files reading and writing
capabilities and also a Matlab to \scilab\ conversion tool. Scilab can
also interact with other major programming languages such as Fortran,
C, C++, Python, Java and TCL/TK \cite{scilab-interop}.  It has a
graphical editor called Xcos, which is similar to Simulink of Matlab. 

\subsection{Downloading and installing \scilab}\label{scilab-installation}
Scilab can be downloaded free of cost from {\tt https://www.scilab.org/}. It is
available for all popular operating systems, including Windows, Linux
and Mac OS X systems.  This book uses Scilab-5.5.2 for demonstrating
the experiments, both on Windows and Linux.

For Windows system, the downloaded file is an executable (.exe) file.
It has to be double clicked to start the installation. All the default
parameters of installation are acceptable.  It has to be noted that
\scilab\ requires internet connectivity during installation on
Windows.  There is an option in the beginning of the installation to
continue offline but it is not recommended.  Scilab can be launched
either from the Start menu or by double clicking on \scilab\ icon
created on the Desktop (if any).

Package managers of Linux do not have the latest versions of Scilab.
As a result, downloading directly from the \scilab\ website, {\tt
            https://www.scilab.org/}, is recommended, as there are differences in
Scilab from version to version.  We expect such problems to be overcome
in the future.  The file downloaded will be in a tar.gz format.  It has
to be extracted before use.  It is a portable version and needs no
installation. Scilab can be launched and used right away.  To launch
\scilab, open a terminal by pressing the Alt+Ctrl+T keys
together. Change the directory where \scilab\ is extracted. Browse
till the {\tt /bin} directory. Type the command {\tt ls} to see a few
\scilab\ files.  Then execute the command {\tt sudo ./scilab}. Note
that \scilab\ needs to be launched with root permissions to be able to
communicate with \arduino. This process is illustrated in
\figref{linux-cd}.
\begin{figure}
      \centering
      \includegraphics[scale=0.5]{\LocSWfig/linux-cd.png}
      \caption{Linux terminal to launch Scilab}
      \label{linux-cd}
\end{figure}

\subsection{Scilab Arduino toolbox}
Scilab, by default, does not have the capability to connect to
Arduino. All such add-on functionalities are added to \scilab\ using
toolboxes. Just like we have different installation binaries of
\scilab\ for Windows and Linux, we have different toolboxes types for
Windows and Linux. The \scilab\ Arduino toolbox can be found inside
the {\tt Origin/tools/windows} or {\tt Origin/tools/linux} directory,
see \fnrefp{fn:file-loc}.  Use the one depending upon
which operating system you are using. The \scilab\ codes for various
experiments mentioned throughout this book can be found in {\tt
            Origin/user-code} directory. The {\tt user-code} directory will have
many sub-directories as per the experiments.

Let us now see how to load the Scilab Arduino toolbox. 
\begin{enumerate}
      \item First launch \scilab. On a Windows system, one may start/launch
            \scilab\ either through the Start menu or by double clicking on the
            shorcut icon created on the Desktop. On a Linux system, one has to
            start \scilab\ through a terminal with root permissions, as
            explained in section \ref{scilab-installation}.
      \item After launching \scilab, first we have to change the working
            directory. To do so, click on the {\tt File} menu and then click on
            the {\tt Change current directory} option as shown in
            \figref{scilab-cd}.
            \begin{figure}
                  \centering
                  \includegraphics[width=\linewidth]{\LocSWfig/change-directory.png}
                  \caption{Changing scilab directory}
                  \label{scilab-cd}
            \end{figure}
      \item Then, one has to browse to the toolbox folder
                  {\tt Origin/tools/windows} or {\tt Origin/tools/linux}, as the case
            may be, and click on, {\tt
                        open}, as shown in \figref{scilab-browse}.
            \begin{figure}
                  \centering
                  \includegraphics[width=\hgfig]{\LocSWfig/browse-directory.png}
                  \caption{Browsing toolbox directory}
                  \label{scilab-browse}
            \end{figure}
      \item After the previous step, the \scilab\ working directory becomes
            the toolbox folder.  See the {\tt file browser} panel on the
            left hand side of the \scilab\ console, see \figref{builder}.  It
            will list out the contents of your current working directory. For a
            check, look for the file {\tt builder.sce}.  If you see this file,
            then you are in the right directory.
      \item Next, type the following command on the \scilab\ console: {\tt
            exec builder.sce} - this will build the toolbox and create a file
                  {\tt loader.sce}. This step has to be executed only the first
            time. Output of this step is illustrated in \figref{builder}.
            \begin{figure}
                  \centering
                  \includegraphics[width=\linewidth]{\LocSWfig/builder.png}
                  \caption{Output of builder.sce}
                  \label{builder}
            \end{figure}
      \item Next, type the command,
            {\tt exec loader.sce} -
            this will load the toolbox. This means all the new functions
            corresponding to the toolbox are all loaded in the workspace. It
            will also make available new Xcos blocks, if any.  The
            output of this command is as shown in \figref{loader}.  If you clear
            the work space for any reason, you will have to execute this command
            once again\footnote{Be careful
                  not to execute the {\tt clear} command.  This will clear the loaded
                  toolbox and you will have to execute the loader.sce file again.}
            \begin{figure}
                  \centering
                  \includegraphics[scale=0.5]{\LocSWfig/loader.png}
                  \caption{Output of loader.sce}
                  \label{loader}
            \end{figure}
\end{enumerate}
The toolbox is now loaded and available for use. 

\subsection{Identifying Arduino communication port number}

Connect \arduino\ board to your computer. On a Windows system, doing
so for the first time will initiate Windows device identification
routine. It may take a while before it finishes assigning a COM port
number to the \arduino\ board.  If Arduino IDE is installed using the
procedure outlined in \secref{arduino-ide}, required USB drivers for
Arduino get installed automatically.  Hence if you have installed the
Arduino IDE, it should not ask for drivers after you connect it.  As
usually Linux systems come with required drivers, the device is
automatically detected by the OS on connection.

Now let us see how to identify the COM port number. For a Windows
system, open the Device Manager. To do so, right click on ``My
Computer'' and choose Properties. The Properties window that will open
will have Device Manager in the list on the left hand side. In the
Device Manager window, look for ``Ports (COM and LPT)''. Double click on
it. It will show you the COM number for \arduino. 

\begin{figure}
      \centering
      \includegraphics[width=\linewidth]{\LocSWfig/device-manager.png}
      \caption{Device Manager in windows}
      \label{dev-mgr}
\end{figure}

The result of the above exercise is shown in \figref{dev-mgr}.  In
this case, the system has detected Arduino with port number 3, which
appears as COM3.  In this book, we have taken the port for
communication as 2 and written code consistent with this assumption.    
As a result, we will now change it to
COM2\footnote{\label{fn:port}It is possible to leave it at whatever
      port number one gets.  It is also possible to choose any number
      between 2 and 99.  In this case, the port number should be
      changed accordingly in the code.  We will point this out throughout
      the book.}.  To change the port number, double click on the port
number. Its properties window will appear. Click on the ``Port
Settings'' tab and then click on ``Advanced'' button as shown in
\figref{com}.

\begin{figure}
      \centering
      \includegraphics[scale=0.5]{\LocSWfig/com-properties.png}
      \caption{COM port properties window}
      \label{com}
\end{figure}

Click on the drop down menu for COM port numbers. Choose the port
number COM2.  On clicking on ``OK'', Windows may warn you that the port
number is already in use. But given that you do not have any other USB
device connected you may force change it. Click on ``OK'' to close
all of the device manager windows. Now, we are set to go ahead with
port number 2. The stress on using port number 2 is just to be
consistent throughout the book. It is mainly for a beginner.

Now, let us see how to identify the port number on a Linux
system. Open a terminal by pressing Alt+Ctrl+T keys together. Then
type the following command and press enter, {\tt ls
            /dev/ttyACM*} -
the output of this command is shown in \figref{linux-port}. It has
detected the Arduino with port number ``ttyACM0''.  The last character
in this string, namely 0, is the port number.  You may get 0 or a
number such as 1 or 2 in your case, for the port number.

\begin{figure}
      \centering
      \includegraphics[scale=0.5]{\LocSWfig/linux-port.png}
      \caption{Port number on Linux terminal}
      \label{linux-port}
\end{figure}

\subsection{Testing Scilab-Arduino toolbox}
\label{sec:testing-scilab-arduino}
Now let us test the functioning of the toolbox. 
\begin{enumerate}
      \item Install Arduino IDE, as explained in \secref{sec:ard-start} and
            launch it.
      \item Read into the Arduino IDE, the firmware \ardref{ard:firmware}.
      \item Using the {\tt Upload} option of the Arduino IDE, load this
            firmware on to the \arduino\ board.
      \item Inside the {\tt Origin/tools} directory, locate a file {\tt
                        test\_firmware.sce}. This file will be used to test whether the
            firmware is properly installed.  This is an important step, as the
            connection between the computer and Arduino breaks down sometimes.
            The Scilab toolbox is unable to identify this difficulty - it has to
            be externally done.  If this difficulty is not identified and
            rectified, one will waste a lot of time and effort trying to debug
            the error.  This test has to be done in case of difficulties.
      \item In the \scilab\ console, type {\tt editor} and press the enter
            key. This will launch the editor. Click on ``File'' menu and choose
            ``Open''. Browse to the directory {\tt Origin/tools} and choose the
            file {\tt test\_firmware.sce}.  It will open
            \sciref{sci:test-firmware}.  
            %The \scilab\ editor with this file open is as shown in \figref{test-code}.
            %   \begin{figure}
            %     \centering
            %     \includegraphics[scale=0.5]{\LocSWfig/test-code.png}
            %     \caption{Scilab code to test toolbox and firmware}
            %     \label{test-code}
            %   \end{figure}
            
      \item If you are using a Windows system and have set your port number
            as COM2, you need not make any changes to the file. Linux users,
            however, will mostly identify the port number as ``ttyACM0''. Hence, 
            they need to change the following line number
            \lstinputlisting[firstline=2,lastline=2]
            {\LocSWchkcode/test_firmware.sce}
            to
            \begin{lstlisting}[style=nonumbers]
  h=open_serial(1,0,115200)
\end{lstlisting}
            
      \item To execute this code, on the menu bar, click on the {\tt
                        Execute}, option. Then choose {\tt File with no echo}. The output
            will appear on the console as shown in \figref{test-console}.
            \begin{figure}
                  \centering
                  \includegraphics[scale=0.5]{\LocSWfig/test-console.png}
                  \caption{Scilab test code output}
                  \label{test-console}
            \end{figure}
            As shown in the figure, we see the response of this code as ``ans = '' and
            ``ok'' three times.  The
            code basically gives some input to Arduino three times and the
            program inside it returns ``ok'' three times.  This code thus confirms
            the working of the Scilab-Arduino toolbox.  The code also confirms
            that the firmware inside the Arduino is correct.  It is alright if
            one or two of the attempts out of three give a blank response.  But
            all the three responses certainly should not be
            blank\footnote{\label{fn:firmware}If this step is unsuccessful,
                  one should check the connections and re-install the firmware}.
\end{enumerate}

Now let us take a look at the various functions facilitated by the
toolbox. The functions provided in the toolbox are as shown in 
\figref{func}. They are basically categorized into four categories:
configuration, digital, analog and motors. These functions will be
explained in detail in the subsequent chapters as and when they are
used.

\begin{figure}
      \centering
      \includegraphics[width=\linewidth]{\LocSWfig/table_functions_crop.pdf}
      \caption{Arduino toolbox functions used in this book}
      \label{func}
\end{figure}

\subsection{Firmware}
\lstset{style=mystyle}
\label{sec:test-firmware-scilab}
\addtocontents{cod}{\protect\addvspace{\codclr}}
We have provided a Scilab code to check whether the firmware has been
properly installed.  That code is listed below.

\begin{scicode}
      \ccaption{A Scilab code to check whether the firmware is
        properly installed or not}{A Scilab code to check whether the
        firmware is properly installed or not.  Available at 
            \LocSWchkbrief{test\_firmware.sce}. Execute this code 
            by following the steps given in section 
            \ref{sec:testing-scilab-arduino}.}
      \label{sci:test-firmware}
      \lstinputlisting{\LocSWchkcode/test_firmware.sce}
\end{scicode}


\section{Xcos}
\label{sec:xcos-start}
Xcos is a graphical editor for \scilab\ \cite{xcos-ref}. Most of the
mathematical manipulations that can be done using \scilab\ scripts,
can be done using Xcos also.  The major advantage of Xcos is the
intuitive interface and easy connectivity across blocks. Xcos even
supports {\tt if else}, {\tt for} and {\tt while} looping which form
an integral part of any programming language. It is possible to code
the entire algorithm using Xcos blocks alone. It is also possible to
read from and write to \scilab\ workspace through Xcos.

\subsection{Downloading, installing and testing}
Xcos comes pre-installed with \scilab. Hence a separate installation
of Xcos is not required. Let us explore the functionalities Xcos has
to offer. Xcos basically provides a graphical interface to \scilab.  

Xcos can be launched from \scilab\ by clicking on the Xcos icon
available on the \scilab\ menu bar. It can also be launched by simply
typing the command {\tt xcos} in the \scilab\ console. When Xcos is
launched, it will open a palette browser.  We have shown this in
\figref{sine-blk}, where we have selected a sine block.  At the time
of launch, Xcos will also open an empty canvas, called untitled
Xcos window.

\begin{figure}
      \centering
      \includegraphics[width=\hgfig]{\LocSWfig/sine-blk.png}
      \caption{Sine generator in palette browser}
      \label{sine-blk}
\end{figure}

% \begin{figure}
% \centering
% \includegraphics[scale=0.5]{\LocSWfig/palette-browser.png}
% \caption{Palette Browser}
% \label{palette}
% \end{figure}


% \begin{figure}
% \centering
% \includegraphics[scale=0.5]{\LocSWfig/untitled-xcos.png}
% \caption{Untitled Xcos window}
% \label{untitled}
% \end{figure}

Palette browser shows all of the available blocks that can be used. It
has been nicely categorized as per the functionality. For example,
blocks which generate signals/values without any input, fall under the
category {\tt Sources}. Similarly, blocks which take signals/values
without giving any output are categorized as {\tt Sinks}. This makes
finding a particular block very easy, specially when one does not know
the name of a block.

The untitled window is the one where one creates the Xcos
code/diagram. The relevant blocks have to be dragged and dropped from
the palette browser to the untitled window. The blocks are then
interconnected and configured as per the simulation, which we will
demonstrate next.

\subsection{Use case}
Let us build a simple Xcos simulation to plot a sine wave. This
simulation requires a sine wave source. It can be found in the {\tt
            Sources} category as shown in \figref{sine-blk}. Drag and drop this
block in the untitled Xcos window. 

Next, we need a block to plot the sine wave. A plotting block can be
found in the {\tt Sinks} category as shown in \figref{plot-blk}. The
name of this block is CSCOPE. Drag and drop this block in the untitled
Xcos window.  On the left hand side, this block has an input port for
data.  It is black in colour, which may not be obvious in a black and
white print out.  The output from the sine block has to be connected
to this port.  At its top side, the CSCOPE block has another input
port, called an event port.  This is red in colour.  This port is used
to synchronise it with event generating devices.

\begin{figure}
      \centering
      \includegraphics[width=\hgfig]{\LocSWfig/plot-blk.png}
      \caption{CSCOPE block in xcos}
      \label{plot-blk}
\end{figure}

As the CSCOPE block has an
input event port, we need a source which generates events. Hence, the
next block that we need is an event generator block and it can be
found in the {\tt Sources} category. This is illustrated in figure
\ref{clk-blk}. The name of this block is CLOCK\_c. Drag and drop this
block in the untitled Xcos window.

\begin{figure}
      \centering
      \includegraphics[width=\hgfig]{\LocSWfig/clock-blk.png}
      \caption{CLOCK\_c block in xcos}
      \label{clk-blk}
\end{figure}

The next step is to interconnect the blocks together. A black color
port can only be connected to other black color port. A black color
port cannot be connected to a red color port and vice versa.  That is,
a data port cannot be connected to an event port.  Linking
two blocks is bit crucial and may need a few attempts before one gets
comfortable. To link two blocks, first click and hold the left mouse
button over the output port of the source block. Without releasing the
mouse button, touch the mouse pointer to the input port of the sink
block. If a connection is possible there, the port will turn
``green''. At this point, release the mouse button and the blocks should
get connected. Follow this procedure and complete the connection as
shown in the \figref{sine-gen}.  Save this file with name {\tt
            sine-generator}.  

\begin{figure}
      \centering
      \includegraphics[width=\smfig]{\LocSWfig/sine-gen.png}
      \caption{Sine generator in Xcos}
      \label{sine-gen}
\end{figure}

Let us simulate this Xcos code. On the menu bar, click on the {\tt
            Simulation} menu and choose {\tt Start}. You will get a graphic
window with a running sine wave as shown in \figref{sine-output}.

\begin{figure}
      \centering
      \includegraphics[width=\lgfig]{\LocSWfig/sine-output.png}
      \caption{Sine generator Xcos output}
      \label{sine-output}
\end{figure}

This is because we are running the simulation using the default
configuration.  We would like a stationary plot.  If the simulation is
still running, go to the Simulation menu and choose Stop.  Double
click on the CSCOPE block. Its properties window will appear as shown
in \ref{cscope-config}. Note the value of the {\tt Refresh period}. It
is by default 30. Click on Ok.

\begin{figure}
      \centering
      \includegraphics[width=\lgfig]{\LocSWfig/cscope-config.png}
      \caption{CSCOPE configuration window}
      \label{cscope-config}
\end{figure}

Next, on the menu bar, click on the {\tt Simulation} menu and choose
      {\tt Setup}. The {\tt Set Parameters} window will open. The first
parameter is {\tt Final integration time}. It decides for how long the
simulation will run. Change it to be equal to the {\tt Refresh period}
of the CSCOPE block.  That is, change it to 30 as shown in
\figref{sim-setup}. Now start the simulation and you will get a static
plot.  Other paramenters of blocks can also be changed. For example,
one may want to change the input amplitude/frequency or change the
plot scales etc. All these are left to the reader to explore.

\begin{figure}
      \centering
      \includegraphics[width=\lgfig]{\LocSWfig/sim-setup.png}
      \caption{Simulation setup window}
      \label{sim-setup}
\end{figure}

Although we have demonstrated a very basic level of Xcos simulation,
this idea can be used for complex processes as well.  Using Xcos, it
is possible to have user-defined blocks. The user can code the
working of the block as a function in \scilab\ script and then call it
from Xcos.  It is also possible to create subsystems.  
One can even read from and write to C binaries.  Xcos comes with
several pre-defined libraries and hence, it is possible to carry out
other kinds of simulation, such as electrical circuit simulation and
basic thermo-hydraulic simulation, for example.  A detailed
explanation and demonstration is beyond the scope for this book.

\subsection{Xcos-Arduino}
The \scilab\ Arduino toolbox not only provides functions to be used in
\scilab\ scripts but also provides with new Arduino-specific blocks. As
shown in \figref{arduino-palette} new Arduino blocks are now available
for use.  Similar to the categorization of the functions, the Xcos
blocks are also categorized as configuration, digital, analog and
motors. Again, it is possible to conduct the experiments only using
Xcos. Xcos codes for every experiment are provided througout the
book. The Arduino blocks can be easily connected to Xcos native
blocks. A detailed block help for every block can be sought by right
clicking on the block and choosing ``Block help''. This is illustrated
in \figref{blk-help}.

\begin{figure}
      \centering
      \includegraphics[width=\lgfig]{\LocSWfig/arduino-palette.png}
      \caption{Palette browser showing Arduino blocks}
      \label{arduino-palette}
\end{figure}

\begin{figure}
      \centering
      \includegraphics[width=\lgfig]{\LocSWfig/xcos-help.png}
      \caption{Xcos block help}
      \label{blk-help}
\end{figure}

%%%%%python description starts
\section{Python}
\label{sec:python-start}
Python is a general-purpose, high-level, remarkably powerful dynamic programming language 
that is used in a wide variety of application domains. Its high-level built in data structures, 
combined with dynamic typing and dynamic binding, make it very attractive for Rapid Application Development, 
as well as for use as a scripting or glue language to connect existing components together. 
Python's simple, easy to learn syntax emphasizes readability and therefore reduces the cost of program maintenance. 
Python supports modules and packages, which encourages program modularity and code reuse. 
The Python interpreter and the extensive standard library are available in source or binary form without 
charge for all major platforms, and can be freely distributed \cite{python-ref}.


\subsection{Downloading and installing Python}\label{python-installation}
Python can be downloaded free of cost from {\tt https://www.python.org/}. It is
available for all popular operating systems, including Windows, Linux
and Mac OS X systems.  This book uses Python 3.5 for demonstrating
the experiments, both on Windows and Linux.

For Windows system, the downloaded file is an executable (.exe) file. One can download 
this executable file (32-bit or 64-bit as per your system) from Python's webpage i.e.,
{\tt https://www.python.org/downloads/windows/}. 
It has to be double clicked to start the installation. All the default parameters of installation
are acceptable. Once Python has completed the downloading and installation process, we can verify whether
Python is installed correctly. For this, we need to open Windows PowerShell using the Start menu. 
Once PowerShell is open, enter {\tt Python -{}-version} to confirm that Python has installed on your machine. 

Apart from Python, we need to install Python Serial Port Extension. 
For this, one can download the
Windows pyserial package from the PyPI website, i.e., {\tt https://pypi.python.org/pypi/pyserial/3.4} and 
install it by double-clicking on the downloaded file. 

On Linux, we can install Python from the terminal. Open a terminal by 
pressing the Alt+Ctrl+T keys together. Then, update the system by executing the 
command {\tt sudo apt-get update}. Next, install Python3.5 by executing the command given below:\\
{\tt sudo apt-get install python3.5}\\
Now, we will install Python Serial Port Extension by executing the commands given below:\\ 
{\tt sudo apt-get install python3-pip \\
pip3 install pyserial}. 

To execute a Python code (or script) on Linux, one should 
navigate the terminal to the directory where the script is located using the {\tt cd} command.
Next, type {\tt python3 SCRIPTNAME.py} in the terminal to execute the script. 

% \begin{enumerate}
%       \item On Ubuntu, Python can be installed from the command line by typing in the
%             following commands:
            
%             \$ sudo apt-get install python3.5 \\
%             This will install python 3.5 as this set of tools are compatible with python3. It also
%             works all other versions of python3.  
%             \$ pip3 install python-serial \\
%             This will install serial package required for communicating with Arduino Uno.
%       \item 1. On Windows,
            
            
%             Download Windows python compiler Self Extracting Archive (.exe)  (32-bit or 64-bit as per your system) from https://www.python.org/downloads/windows/
            
%             Download Windows pyserial package .exe from https://pypi.python.org/pypi/pyserial/3.4.
%             After downloading run the .exe file and follow the instructions.
            
%             Now that we have Python installed, open the terminal by typing 'Ctrl+Alt+T'.
%             Enter the command 'python'. This opens the Python REPL (Read Eval Print Loop).
%             Python files have .py extension and .py files can be executed by typing "python3 filename.py"
%             on the python terminal. Please visit the link https://github.com/manasdas17/Python3-Arduino for 
%             the library. 
            
% \end{enumerate}

\subsection{Firmware}
\lstset{style=mystyle}
\label{sec:test-firmware-python}
\addtocontents{pyd}{\protect\addvspace{\codclr}}
We have provided a Python code to check whether the firmware has been
properly installed. That code is listed below. 
Please ensure that you have uploaded the Arduino firmware given in \ardref{ard:firmware} on the \arduino\ board.


\begin{pycode}
      \pcaption{A Python code to check whether the firmware is properly installed
            or not}{A Python code to check whether the firmware is properly installed
            or not.  Available at 
            \LocFIMpybrief{led-blue.py}. 
            Execute this code by following the steps given in section \ref{python-installation}.}
      \label{py:test-firmware}
      \lstinputlisting{\LocFIMpycode/led-blue.py}
\end{pycode}






%%%%%python description ends

%%%%%%Julia descrition starts

\section{Julia}
\label{sec:julia-start}
Julia is a high-level, high-performance dynamic programming language for
technical computing, with syntax that is familiar to users of other technical
computing environments. It provides a sophisticated compiler, distributed
parallel execution, numerical accuracy, and an extensive mathematical function
library. This is a set of tools that provide functionality in Julia, to program the
\arduino. 

\subsection{Downloading and installing Julia}\label{julia-installation}
\begin{enumerate}
\item On Ubuntu, Julia can be installed from the command line by typing in the 
following commands:

\$ sudo add-apt-repository ppa:staticfloat/juliareleases \\
\$ sudo add-apt-repository ppa:staticfloat/julia-deps \\
\$ sudo apt-get update \\
\$ sudo apt-get install julia \\
\item 1. On Windows,


Download Windows Self Extracting Archive (.exe)  (32-bit or 64-bit as per your system) from http://julialang.org/downloads/

\begin{enumerate}
      \item Run the .exe file and follow the instructions. Note the location of the extracted files and folders
      \item Download 'Git Bash' from https://git-scm.com/download/win
      \item Open Git Bash, in the prompt type:
            git config --global url.https://github.com/.insteadOf git://github.com/
            
      \item Download the Editor for Julia, 'Atom' from https://atom.io/
      \item Run AtomSetup.exe and follow the instructions to install Atom on windows
      \item Open Atom>File>Settings>Install
            In the search engine, search for the package 'uber-juno'
            Click on 'Install'
\end{enumerate}

Possible errors:

\begin{enumerate}

      \item Juno could not be found
            Solution: Open Atom>File>Settings>Packages>julia-client:Settings>Julia path
            Change the path to $\backslash$....$\backslash$Julia-0.4.5$\backslash$bin$\backslash$julia.exe
            
      \item Atoms.jl error\\
            Solution: Open Julia terminal\\
            in the prompt, type:\\
            julia>Pkg.status()\\
            julia>Pkg.build()\\
            8. Go to https://github.com/cshekhard/SerialPorts.jl
            Click on 'Clone or Download' and copy the link
            
            9. Open Julia terminal and in the prompt, type:
            julia>Pkg.clone("paste link")
            
      \item PyCall error
            This error may occur if the Python installed on your system is the 32 bit version, whereas the Julia that you just installed is the 64 bit version.
            Sol: In Julia terminal, type:
            julia> ENV["PYTHON"]="";Pkg.build ("PyCall")
            Let the packages download and install.
\end{enumerate}

Julia should work fine now.


\item Other platform specific instructions can be found at \\ 
http://www.julialang.org/downloads/platform.html 

\item You can also clone the source repository and compile/build it to get the  
latest development version of julia. However this may prove to be unstable.
https://github.com/JuliaLang/julia/tree/master 

\end {enumerate}
%This process is illustrated in
%\figref{linux-cd}.
%\begin{figure}
%\centering
%\includegraphics[scale=0.5]{\LocSWfig/linux-cd.png}
%\caption{Linux terminal to launch Scilab}
%\label{linux-cd}
%\end{figure}

\subsection{Downloading and installing Juno IDE}\label{juno-ide-installation}

Juno is a powerful IDE for the Julia language, hosted as a package on the Atom IDE. Installing it also installs some common Julia packages such as PyCall and Plots.The package is hosted at \\
https://github.com/JunoLab/uber-juno/blob/master/setup.md 

Download, install and open Atom. If you have it already, make sure it's up to date (version 1.7+).
In Atom, go to Settings and go to the "Install" panel.
Type uber-juno into the search box and hit enter. Click the install button on the package of the same name.
Atom will then set up Juno for you, installing the required Atom and Julia packages.

Now that you have Julia installed,open the terminal by typing 'Ctrl+Alt+T'.
Enter the command 'julia'. This opens the Julia REPL (Read Eval Print Loop)
as shown in
\figref{julia-terminal}.
\begin{figure}
      \centering
      \includegraphics[scale=0.5]{\LocSWfig/julia-terminal.png}
      \caption{Linux terminal to launch Julia}
      \label{julia-terminal}
\end{figure}

Julia filenames have the extension '.jl' . There are several ways to run a julia file:

\begin{enumerate}
      \item By clicking the Run button in the above environment. This runs the file as a whole and prints any output on the console. Pressing 'Ctrl$+$Enter' evaluates and runs only the line on which the cursor is present.
      \item In the julia REPL: Typing 'include("filename.jl")'' at the julia prompt runs the file. Note that this works only if the file is in the present working directory. Otherwise the entire path to the file should be mentioned in the include command. Alternatively, you can change the present directory to the one in which the file is available by typing ';'.
      \item In the terminal: Navigate to the directory with the file and type command 'julia filename.jl'
\end{enumerate}


After setting up the JUNO IDE as a plugin in the Atom editor, launching Atom will result in something like this:

\figref{atom-lunch}.
\begin{figure}
      \centering
      \includegraphics[scale=0.2]{\LocSWfig/atom-lunch.png}
      \caption{Linux Atom Text Editor}
      \label{atom-lunch}
\end{figure}

\subsection{Julia Arduino toolbox}
Registered packages (those which are listed in the 'packages' section on the julia site) can be added by the command 'Pkg.add()' command, for e.g.

julia> Pkg.add("Pyplot").

All julia packages are GitHub repositories which can be cloned via the appropriate protocol. Unregistered packages which cannot be added as above can be added using Pkg.clone() command and the git clone url. For example,

julia> Pkg.clone("https://github.com/cshekhard/MyPackage.jl.git") 

Care should be taken to use the appropriate protocol - git:// or https:// (if you are behind a firewall). 

This tool makes use of a slightly modified version of the package 'SerialPorts' originally developed by Steve Kelly (sjkelly) - https://github.com/sjkelly/SerialPorts.jl.git 

Let us now see how to load the Julia Arduino toolbox. 


To install:
\begin {itemize}
\item julia> Pkg.clone ("https://github.com/cshekhard/SerialPorts.jl.git")

Make sure the package is installed by checking Pkg.status() again.
After a package is installed, it is imported into the current julia session with the keyword 'using'

\item julia> using SerialPorts
The toolbox is now loaded and available for use. 

\end{itemize}

All the codes or .jl files must be saved in a folder that contains the file 'ArduinoTools.jl'
This is because when we use include("ArduinoTools.jl"), the function 'include' searches for 'ArduinoTools.jl' only in that directory where our program file is saved.




\subsection{Using the serial port package}

The generic functions available with this package are:
\begin{enumerate}
      \item write() - To write data into serial port. Returns the number of bytes written.
            It has three methods:
            \begin{enumerate}
                  \item write(ser::SerialPort,data::UInt8) :
                        To write a single Integer byte (0 to 255).
                  \item write(ser::SerialPort,data::ASCIIString) :
                        To write an ASCIIString into the serial port.
                  \item write(ser::SerialPort,data::UTF8String) :
                        To write a Unicode String into the serial port. Available only in the
                        modified package.
            \end{enumerate}
            
      \item read(ser::SerialPort,n::Integer) - To read n bytes of data from the serial
            port. Returns the data read as a String.
            
      \item readavailable(ser::SerialPort) - To read all available bytes from the serial
            port. Returns the data read as a String. 
            
      \item nb\_available(ser::SerialPort) - Returns the number of bytes waiting to be read
            in the input buffer.
            
      \item open(ser::SerialPort) and close(ser::SerialPort) - To open and close a
            pre-initialized serial port.   
            
\end{enumerate}

All the aforementioned functions have identical analogues defined in
the ArduinoTools module:  Write(), Read(), ReadAvailable(),
nbAvailable(), openSerial() and closeSerial(). Hence, importing only
the ArduinoTools module suffices for using any function. Using this
basic set of functions, the user can define other functions to operate
upon the Arduino.


\subsection{Firmware}
\lstset{style=mystyle}
\label{sec:test-firmware-julia}
\addtocontents{cod}{\protect\addvspace{\codclr}}
We have provided a Julia code to check whether the firmware has been
properly installed.  That code is listed below.

\begin{juliacode}
      \jcaption{A Julia code to check whether the firmware is properly installed
            or not}{A Julia code to check whether the firmware is properly installed
            or not.  Available at
            \LocFIMjuliabrief{test\_firmware.jl}.}
      \label{julia:test-firmware}
      \lstinputlisting{\LocFIMjuliacode/test_firmware.jl}
\end{juliacode}




%%%%%%%%OpenModelica description starts


\section{OpenModelica}
\label{sec:OpenModelica-start}
OpenModelica is a free and open source environment based on the Modelica modeling language for simulating, optimizing and analyzing complex
dynamic systems.OpenModelica is a powerful tool that can be used to design and
simulate complete control systems.The toolbox 'OpenModelica-Arduino'
enables the interfacing of Arduino with OpenModelica by calling a set of c functions
from OpenModelica.   




\subsection{Downloading and installing OpenModelica}\label{OpenModelica-installation}

On Ubuntu, OpenModelica can be installed from the command line by typing in the 
following commands:

\$ for deb in deb deb-src; do echo "$deb http://build.openmodelica.org/apt $'lsb\_release -cs'$ stable$"; done | sudo tee /etc/apt/sources.list.d/openmodelica.list \\
\$ wget -q http://build.openmodelica.org/apt/openmodelica.asc -O- | sudo apt-key add -  \\
\$ apt-key fingerprint
\$ sudo apt-get update \\
\$ sudo apt-get install openmodelica \\
\$ sudo apt install omlib-.* \\
On Windows,
download Windows Self Extracting Archive (.exe)  (32-bit or 64-bit as per your system) from https://openmodelica.org/download/download-windows. Run the .exe file and follow the instructions to complete the
installation.

%This process is illustrated in
%\figref{linux-cd}.
%\begin{figure}
%\centering
%\includegraphics[scale=0.5]{\LocSWfig/linux-cd.png}
%\caption{Linux terminal to launch Scilab}
%\label{linux-cd}
%\end{figure}

\subsection{Executing OpenModelica}\label{OpenModelica-code-execution}
OpenModelica filenames have the extension '.mo' . There are several ways to run an OpenModelica file.
But in our case, we will execute the code by OMEdit.

\begin{itemize}
      \item On Ubuntu systems by pressing 'Ctrl$+$Alt$+$T', a new terminal will come up.
      \item Then type 'OMEdit' to start OMEdit connection editor. On Widows, OMEdit can be started by clicking on 'OMEdit Connection Editor' icon.
      \item Now from File -> Load Model/Libray File(s) browse to the folder having files having .mo extensions and open it on the OMEdit Editor.
      \item But as OMEdit is a graphical editor so at the first instance, the texts on the code won't be visible on it.
      \item To see the codes, click on the 'Text View' icon.
\end{itemize}



\subsection{OpenModelica Arduino toolbox}

\begin{itemize}
      \item To install and run the OpenModelica Arduino toolbox, browse to
            File -> Open Model and browse to Arduino.mo in the OpenModelica-Arduino-Windows/Linux library and load it.
      \item Click on the '$+$' on left side of Arduino on OMEdit
      \item Then click on '$+$' on the left side of SerialCommunication
      \item Now, click on '$+$' on the left side next to Examples icon. Then you will come to see all
            the experiments available on this toolbox. You can perform any experiment by clicking on the experiment
            name and then execute the 'Simulate' button. 
      \item But don't forget to load the arduino\_firmware used for scilab-arduino to the Arduino board.
\end{itemize}



%%%%%%%%OpenModelica description ends

% \begin{enumerate}

% \item THE connectBoard FUNCTION
%       connectBoard(baudrate::Int64)

%     \begin{enumerate}
%         \item Function that opens the first serial port among the names returned by the 
%               list\_serialports() function, with the specified baud rate.
%         \item Works fine as long as there is no other active serial port in use.
%         \item If there are multiple active serial ports, use the openSerial() function as 
%               follows: openSerial(" <portname> ",<baudrate>)
%         \item Usage: connectBoard(115200)
%     \end{enumerate}

% \item THE pinMode FUNCTION

%       pinMode(file\_des::SerialPorts.SerialPort, pin\_no::Int64, mode::ASCIIString)


%  A function to set the mode of a digital pin as INPUT or OUTPUT \\
%  Parameters:
%  \begin{enumerate}
%  \item file\_des: file descriptor (SerialPort object)
%   \item pin\_no: pin number 
%   \item mode: "INPUT" or "OUTPUT" 
% Similar to the pinMode function in Arduino.
% Usage example : pinMode(ser,9,"OUTPUT")

% \end{enumerate}



% \item THE digiWrite FUNCTION


% digiWrite(file\_des::SerialPorts.SerialPort , pin\_no::Int64 , val::Int64)
% Function to send a digital signal (High(1) or Low(0)) to a connected Arduino
%   board.
%  Parameters:

% \begin{enumerate}

% \item  file\_des: file descriptor (SerialPort object)
%  \item pin\_no: pin number
%   \item val: Digital value to be sent to pin, 0 (for LOW/0V) or 1 (for HIGH/5V)
% \end{enumerate}

% \item THE digiRead FUNCTION



% digiRead(file\_des::SerialPorts.SerialPort, pin\_no::Int64)

%  Function to read the digital value on a pin. Returns 0 or 1 (Integer)\\
%  Parameters:
%  \begin{enumerate}
% \item  file\_des: file descriptor (SerialPort object)
%  \item  pin\_no: pin number
% similar to digitlRead function in Arduino.
% Usage example: digiRead(ser,9)
% \end{enumerate}

% \item THE analogWrite FUNCTION


% analogWrite(file\_des::SerialPorts.SerialPort, pin\_no::Int64, val::Int64)

% - Function to send an analog signal to a connected Arduino board.\\
% - Parameters:
% \begin{enumerate}
%   \item file\_des: SerialPort object
%   \item pin\_no: pin number (3,5,6,9,10 or 11 only in case of Arduino UNO)
%   \item val: The value between 0-255 sent to the pin which will then be scaled to 
%        0V-5V.
%   \end{enumerate}

% \item THE analogRead FUNCTION:


% analogRead(file\_des::SerialPorts.SerialPort , pin\_no::Int64)

% - Function to read analog signal from a pin on Arduino board. Returns a value 
%   between 0-1023.\\
% - Parameters: 
% \begin{enumerate}
%   \item file\_des: SerialPort object
%      \item pin\_no: pin number

%   \end{enumerate}


% \item THE DCMotorSetup FUNCTION

% DCMotorSetup(file\_des::SerialPorts.SerialPort, driver\_type::Int64,  
%              motor\_no::Int64, pin1::Int64, pin2::Int64)

% - Function to initialize and setup a DC motor.\\
% - Parameters:

% \begin{enumerate}

%  \item file\_des: SerialPort object
%   \item driver\_type: 1 for adafruit
%                    2 for L298
%                    3 for L293
%   \item motor\_no: A number for identifying the motor – 1,2,3 or 4.

% \end{enumerate}

% \item THE DCMotorRun FUNCTION

% DCMotorRun(file\_des::SerialPorts.SerialPort, motor\_no::Int64, speed::Int64)

% - Function to run the motor at a desired speed, clockwise or anticlockwise.\\
% - Parameters:

% \begin{enumerate}


%   \item file\_des: SerialPort object
%   \item motor\_no: The ID number given to the motor in the setup function
%   \item speed: Ranging from -255 to 255, change in sign will reverse the direction
% \end{enumerate}

% \item THE DCMotorRelease FUNCTION 
% DCMotorRelease(file\_des::SerialPorts.SerialPort, motor\_no::Int64)

% - Function to release a DC motor from the Arduino board i.e revoke the
%   parameters from the setup function \\
% - Parameters:
% \begin{enumerate}



%  \item file\_des: SerialPort object
%   \item motor\_no: ID number of the motor from setup.

% \end{enumerate}


% \item THE ServoAttach FUNCTION

% ServoAttach(file\_des::SerialPorts.SerialPort, servo\_no::Int64)

% - Function to attach and initialize a servomotor \\
% - Parameters:

% \begin{enumerate}

%   \item file\_des: SerialPort object
%   \item servo\_no: 1 (for servo attached to pin 9) and 2 (for servo attached to 
%                 pin 10)
% \end{enumerate}

% \item THE ServoMove FUNCTION

% - Function to move the rotating part of motor to a specific angular position\\
% - Parameters:

% \begin{enumerate}

%   \item file\_des: SerialPort object
%   \item servo\_no: 1 (pin 9) or 2 (pin 10)
%   \item val: angular position from 0 – 180 (in degrees)
% \end{enumerate}

% \item THE ServoDetach FUNCTION

% ServoDetach(file\_des::SerialPorts.SerialPort, servo\_no::Int64)

% - Function to detach the servomotor from arduino board \\
% - Parameters:
% \begin{enumerate}

% \item  file\_des: Serialport object
%  \item  servo\_no: 1 or 2
% \end{enumerate}



% \end{enumerate}



%% \section{Code}
%% \lstset{style=mystyle}
%% \label{sec:hw-arduino}
%% \addtocontents{ard}{\protect\addvspace{\codclr}}
 

%python code
%%%%%%%%%end python code

%%%%%begin julia code
%%%%%end julia

%%%%%begin OpenModelica code

\subsection{Firmware}
\lstset{style=mystyle}
\label{sec:test-firmware-OpenModelica}
\addtocontents{cod}{\protect\addvspace{\codclr}}
We have provided an OpenModelica code to check whether the firmware has been
properly installed.  That code is listed below.

\begin{OpenModelicacode}
      \mcaption{An OpenModelica code to check whether the firmware is properly installed
            or not}{An OpenModelica code to check whether the firmware is properly installed
            or not.  Available at
            \LocFIMOpenModelicabrief{test\_firmware.mo}.}
      \label{OpenModelica:test-firmware}
      \lstinputlisting{\LocFIMOpenModelicacode/test_firmware.mo}
\end{OpenModelicacode}

%%%%%end OpenModelicamo
